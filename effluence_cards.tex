\documentclass[parskip]{scrartcl}
\usepackage[margin=10mm]{geometry}
\usepackage{tikz}
\usepackage{pifont}
\usepackage{graphicx}
%\usepackage{fontawesome}
%\newfontfamily{\FA}{FontAwesome Regular}

\begin{document}

\pgfmathsetmacro{\cardroundingradius}{4mm}
\pgfmathsetmacro{\striproundingradius}{3mm}
\pgfmathsetmacro{\cardwidth}{6}
\pgfmathsetmacro{\cardheight}{9}
\pgfmathsetmacro{\stripwidth}{1.2}
\pgfmathsetmacro{\strippadding}{0.1}
\pgfmathsetmacro{\textpadding}{0.3}
\pgfmathsetmacro{\ruleheight}{0.1}
\newcommand{\stripfontsize}{\Huge}
\newcommand{\captionfontsize}{\LARGE}
%\newcommand{\captionfontsize}{\Large}
\newcommand{\textfontsize}{\large}

% Font Awesome icons

%\newcommand{\event_icon}{ \symbol{"\f0e7"} }
%\newcommand{\microbe_icon}{ \symbol{"\f0c3"} }
%\newcommand{\infection_icon}{ \symbol{"\f0fa"} }
%\newcommand{\plasmid_icon}{ \symbol{"\f085"} }
\begin{tabular}{c c c}

\begin{tikzpicture}

	\draw[red,fill=red] (5.2,.9) circle (4ex);
	\draw[red,fill=green] (3.7,.9) circle (4ex); 
	\node at (3.7,0.9) {\LARGE \bfseries 0}; 
	\node at (5.2,0.9) {\LARGE \bfseries -2};
    \draw[rounded corners=\cardroundingradius] (0,0) rectangle (\cardwidth,\cardheight);
    \fill[orange,rounded corners=\striproundingradius] (\strippadding,\strippadding) rectangle (\strippadding+\stripwidth,\cardheight-\strippadding) node[rotate=90,above left,black,font=\stripfontsize] {Microbe \rotatebox[origin=c]{-90}{\ding{49}}};
    \node[text width=(\cardwidth-\strippadding-\stripwidth-2*\textpadding)*1cm,below right,inner sep=0] at (\strippadding+\stripwidth+\textpadding,\cardheight-\textpadding) 
    {   {\captionfontsize \textbf{Pathogen}}\\ 
        {\textfontsize \textit{Clostridium botulin}}\\
        \tikz{\fill (0,0) rectangle (\cardwidth-\strippadding-\stripwidth-2*\textpadding,\ruleheight);}\\
        {\small (text here)}\\
        {\small \small Resistance: \begin{itemize}
\item Ampicillin
\item Tetracycline
\end{itemize}
}
        \rule{4cm}{0.4pt}
        {\small \small \textit{This microbe is responsible for botulism, a kind of food poisoning}}\\
        
    };
\end{tikzpicture}

&

\begin{tikzpicture}

	\draw[red,fill=red] (5.2,.9) circle (4ex);
	\draw[red,fill=green] (3.7,.9) circle (4ex); 
	\node at (3.7,0.9) {\LARGE \bfseries 0}; 
	\node at (5.2,0.9) {\LARGE \bfseries -2};
    \draw[rounded corners=\cardroundingradius] (0,0) rectangle (\cardwidth,\cardheight);
    \fill[orange,rounded corners=\striproundingradius] (\strippadding,\strippadding) rectangle (\strippadding+\stripwidth,\cardheight-\strippadding) node[rotate=90,above left,black,font=\stripfontsize] {Microbe \rotatebox[origin=c]{-90}{\ding{49}}};
    \node[text width=(\cardwidth-\strippadding-\stripwidth-2*\textpadding)*1cm,below right,inner sep=0] at (\strippadding+\stripwidth+\textpadding,\cardheight-\textpadding) 
    {   {\captionfontsize \textbf{Pathogen}}\\ 
        {\textfontsize \textit{Clostridium botulin}}\\
        \tikz{\fill (0,0) rectangle (\cardwidth-\strippadding-\stripwidth-2*\textpadding,\ruleheight);}\\
        {\small (text here)}\\
        {\small \small Resistance: \begin{itemize}
\item Ampicillin
\item Tetracycline
\end{itemize}
}
        \rule{4cm}{0.4pt}
        {\small \small \textit{This microbe is responsible for botulism, a kind of food poisoning}}\\
        
    };
\end{tikzpicture}

&

\begin{tikzpicture}

	\draw[red,fill=red] (5.2,.9) circle (4ex);
	\draw[red,fill=green] (3.7,.9) circle (4ex); 
	\node at (3.7,0.9) {\LARGE \bfseries 0}; 
	\node at (5.2,0.9) {\LARGE \bfseries -2};
    \draw[rounded corners=\cardroundingradius] (0,0) rectangle (\cardwidth,\cardheight);
    \fill[orange,rounded corners=\striproundingradius] (\strippadding,\strippadding) rectangle (\strippadding+\stripwidth,\cardheight-\strippadding) node[rotate=90,above left,black,font=\stripfontsize] {Microbe \rotatebox[origin=c]{-90}{\ding{49}}};
    \node[text width=(\cardwidth-\strippadding-\stripwidth-2*\textpadding)*1cm,below right,inner sep=0] at (\strippadding+\stripwidth+\textpadding,\cardheight-\textpadding) 
    {   {\captionfontsize \textbf{Pathogen}}\\ 
        {\textfontsize \textit{Clostridium dificile}}\\
        \tikz{\fill (0,0) rectangle (\cardwidth-\strippadding-\stripwidth-2*\textpadding,\ruleheight);}\\
        {\small When this species is in play you cannot play any "beneficial only" microbes}\\
        {\small \small Resistance: \begin{itemize}
\item Kanamycin
\item Tetracycline
\end{itemize}
}
        \rule{4cm}{0.4pt}
        {\small \small \textit{Antiobiotic-resistant C. dificile is an increasing problem in hospitals}}\\
        
    };
\end{tikzpicture}

\\

\begin{tikzpicture}

	\draw[red,fill=red] (5.2,.9) circle (4ex);
	\draw[red,fill=green] (3.7,.9) circle (4ex); 
	\node at (3.7,0.9) {\LARGE \bfseries 0}; 
	\node at (5.2,0.9) {\LARGE \bfseries -2};
    \draw[rounded corners=\cardroundingradius] (0,0) rectangle (\cardwidth,\cardheight);
    \fill[orange,rounded corners=\striproundingradius] (\strippadding,\strippadding) rectangle (\strippadding+\stripwidth,\cardheight-\strippadding) node[rotate=90,above left,black,font=\stripfontsize] {Microbe \rotatebox[origin=c]{-90}{\ding{49}}};
    \node[text width=(\cardwidth-\strippadding-\stripwidth-2*\textpadding)*1cm,below right,inner sep=0] at (\strippadding+\stripwidth+\textpadding,\cardheight-\textpadding) 
    {   {\captionfontsize \textbf{Pathogen}}\\ 
        {\textfontsize \textit{Clostridium dificile}}\\
        \tikz{\fill (0,0) rectangle (\cardwidth-\strippadding-\stripwidth-2*\textpadding,\ruleheight);}\\
        {\small When this species is in play you cannot play any "beneficial only" microbes}\\
        {\small \small Resistance: \begin{itemize}
\item Kanamycin
\item Tetracycline
\end{itemize}
}
        \rule{4cm}{0.4pt}
        {\small \small \textit{Antiobiotic-resistant C. dificile is an increasing problem in hospitals}}\\
        
    };
\end{tikzpicture}

&

\begin{tikzpicture}

	\draw[red,fill=red] (5.2,.9) circle (4ex);
	\draw[red,fill=green] (3.7,.9) circle (4ex); 
	\node at (3.7,0.9) {\LARGE \bfseries 0}; 
	\node at (5.2,0.9) {\LARGE \bfseries -2};
    \draw[rounded corners=\cardroundingradius] (0,0) rectangle (\cardwidth,\cardheight);
    \fill[orange,rounded corners=\striproundingradius] (\strippadding,\strippadding) rectangle (\strippadding+\stripwidth,\cardheight-\strippadding) node[rotate=90,above left,black,font=\stripfontsize] {Microbe \rotatebox[origin=c]{-90}{\ding{49}}};
    \node[text width=(\cardwidth-\strippadding-\stripwidth-2*\textpadding)*1cm,below right,inner sep=0] at (\strippadding+\stripwidth+\textpadding,\cardheight-\textpadding) 
    {   {\captionfontsize \textbf{Pathogen}}\\ 
        {\textfontsize \textit{Yersinia pestis (plague)}}\\
        \tikz{\fill (0,0) rectangle (\cardwidth-\strippadding-\stripwidth-2*\textpadding,\ruleheight);}\\
        {\small (text here)}\\
        {\small \small Resistance: \begin{itemize}
\item Kanamycin
\end{itemize}
}
        \rule{4cm}{0.4pt}
        {\small \small \textit{This is the microbe responsible for bubonic plague}}\\
        
    };
\end{tikzpicture}

&

\begin{tikzpicture}

	\draw[red,fill=red] (5.2,.9) circle (4ex);
	\draw[red,fill=green] (3.7,.9) circle (4ex); 
	\node at (3.7,0.9) {\LARGE \bfseries 0}; 
	\node at (5.2,0.9) {\LARGE \bfseries -2};
    \draw[rounded corners=\cardroundingradius] (0,0) rectangle (\cardwidth,\cardheight);
    \fill[orange,rounded corners=\striproundingradius] (\strippadding,\strippadding) rectangle (\strippadding+\stripwidth,\cardheight-\strippadding) node[rotate=90,above left,black,font=\stripfontsize] {Microbe \rotatebox[origin=c]{-90}{\ding{49}}};
    \node[text width=(\cardwidth-\strippadding-\stripwidth-2*\textpadding)*1cm,below right,inner sep=0] at (\strippadding+\stripwidth+\textpadding,\cardheight-\textpadding) 
    {   {\captionfontsize \textbf{Pathogen}}\\ 
        {\textfontsize \textit{Yersinia pestis (plague)}}\\
        \tikz{\fill (0,0) rectangle (\cardwidth-\strippadding-\stripwidth-2*\textpadding,\ruleheight);}\\
        {\small (text here)}\\
        {\small \small Resistance: \begin{itemize}
\item Kanamycin
\end{itemize}
}
        \rule{4cm}{0.4pt}
        {\small \small \textit{This is the microbe responsible for bubonic plague}}\\
        
    };
\end{tikzpicture}

\\

\begin{tikzpicture}

	\draw[red,fill=red] (5.2,.9) circle (4ex);
	\draw[red,fill=green] (3.7,.9) circle (4ex); 
	\node at (3.7,0.9) {\LARGE \bfseries 0}; 
	\node at (5.2,0.9) {\LARGE \bfseries -2};
    \draw[rounded corners=\cardroundingradius] (0,0) rectangle (\cardwidth,\cardheight);
    \fill[orange,rounded corners=\striproundingradius] (\strippadding,\strippadding) rectangle (\strippadding+\stripwidth,\cardheight-\strippadding) node[rotate=90,above left,black,font=\stripfontsize] {Microbe \rotatebox[origin=c]{-90}{\ding{49}}};
    \node[text width=(\cardwidth-\strippadding-\stripwidth-2*\textpadding)*1cm,below right,inner sep=0] at (\strippadding+\stripwidth+\textpadding,\cardheight-\textpadding) 
    {   {\captionfontsize \textbf{Pathogen}}\\ 
        {\textfontsize \textit{Salmonella enterica}}\\
        \tikz{\fill (0,0) rectangle (\cardwidth-\strippadding-\stripwidth-2*\textpadding,\ruleheight);}\\
        {\small (text here)}\\
        {\small \small Resistance: \begin{itemize}
\item Ampicillin
\end{itemize}
}
        \rule{4cm}{0.4pt}
        {\small \small \textit{A common souce of food poisoning, often associated with poultry}}\\
        
    };
\end{tikzpicture}

&

\begin{tikzpicture}

	\draw[red,fill=red] (5.2,.9) circle (4ex);
	\draw[red,fill=green] (3.7,.9) circle (4ex); 
	\node at (3.7,0.9) {\LARGE \bfseries 0}; 
	\node at (5.2,0.9) {\LARGE \bfseries -2};
    \draw[rounded corners=\cardroundingradius] (0,0) rectangle (\cardwidth,\cardheight);
    \fill[orange,rounded corners=\striproundingradius] (\strippadding,\strippadding) rectangle (\strippadding+\stripwidth,\cardheight-\strippadding) node[rotate=90,above left,black,font=\stripfontsize] {Microbe \rotatebox[origin=c]{-90}{\ding{49}}};
    \node[text width=(\cardwidth-\strippadding-\stripwidth-2*\textpadding)*1cm,below right,inner sep=0] at (\strippadding+\stripwidth+\textpadding,\cardheight-\textpadding) 
    {   {\captionfontsize \textbf{Pathogen}}\\ 
        {\textfontsize \textit{Salmonella enterica}}\\
        \tikz{\fill (0,0) rectangle (\cardwidth-\strippadding-\stripwidth-2*\textpadding,\ruleheight);}\\
        {\small (text here)}\\
        {\small \small Resistance: \begin{itemize}
\item Ampicillin
\end{itemize}
}
        \rule{4cm}{0.4pt}
        {\small \small \textit{A common souce of food poisoning, often associated with poultry}}\\
        
    };
\end{tikzpicture}

&

\begin{tikzpicture}

	\draw[red,fill=red] (5.2,.9) circle (4ex);
	\draw[red,fill=green] (3.7,.9) circle (4ex); 
	\node at (3.7,0.9) {\LARGE \bfseries 2}; 
	\node at (5.2,0.9) {\LARGE \bfseries -1};
    \draw[rounded corners=\cardroundingradius] (0,0) rectangle (\cardwidth,\cardheight);
    \fill[orange,rounded corners=\striproundingradius] (\strippadding,\strippadding) rectangle (\strippadding+\stripwidth,\cardheight-\strippadding) node[rotate=90,above left,black,font=\stripfontsize] {Microbe \rotatebox[origin=c]{-90}{\ding{49}}};
    \node[text width=(\cardwidth-\strippadding-\stripwidth-2*\textpadding)*1cm,below right,inner sep=0] at (\strippadding+\stripwidth+\textpadding,\cardheight-\textpadding) 
    {   {\captionfontsize \textbf{Opportunistic}}\\ 
        {\textfontsize \textit{Lactobacillus reuteri }}\\
        \tikz{\fill (0,0) rectangle (\cardwidth-\strippadding-\stripwidth-2*\textpadding,\ruleheight);}\\
        {\small Synthesizes vitamin B12}\\
        {\small \small Not resistant
}
        \rule{4cm}{0.4pt}
        {\small \small \textit{Humans are unable to synthesize this vitamin alone}}\\
        
    };
\end{tikzpicture}

\end{tabular}
\cleardoublepage\begin{tabular}{c c c}

\begin{tikzpicture}

	\draw[red,fill=red] (5.2,.9) circle (4ex);
	\draw[red,fill=green] (3.7,.9) circle (4ex); 
	\node at (3.7,0.9) {\LARGE \bfseries 2}; 
	\node at (5.2,0.9) {\LARGE \bfseries -1};
    \draw[rounded corners=\cardroundingradius] (0,0) rectangle (\cardwidth,\cardheight);
    \fill[orange,rounded corners=\striproundingradius] (\strippadding,\strippadding) rectangle (\strippadding+\stripwidth,\cardheight-\strippadding) node[rotate=90,above left,black,font=\stripfontsize] {Microbe \rotatebox[origin=c]{-90}{\ding{49}}};
    \node[text width=(\cardwidth-\strippadding-\stripwidth-2*\textpadding)*1cm,below right,inner sep=0] at (\strippadding+\stripwidth+\textpadding,\cardheight-\textpadding) 
    {   {\captionfontsize \textbf{Opportunistic}}\\ 
        {\textfontsize \textit{Lactobacillus reuteri }}\\
        \tikz{\fill (0,0) rectangle (\cardwidth-\strippadding-\stripwidth-2*\textpadding,\ruleheight);}\\
        {\small Synthesizes vitamin B12}\\
        {\small \small Not resistant
}
        \rule{4cm}{0.4pt}
        {\small \small \textit{Humans are unable to synthesize this vitamin alone}}\\
        
    };
\end{tikzpicture}

&

\begin{tikzpicture}

	\draw[red,fill=red] (5.2,.9) circle (4ex);
	\draw[red,fill=green] (3.7,.9) circle (4ex); 
	\node at (3.7,0.9) {\LARGE \bfseries 2}; 
	\node at (5.2,0.9) {\LARGE \bfseries -1};
    \draw[rounded corners=\cardroundingradius] (0,0) rectangle (\cardwidth,\cardheight);
    \fill[orange,rounded corners=\striproundingradius] (\strippadding,\strippadding) rectangle (\strippadding+\stripwidth,\cardheight-\strippadding) node[rotate=90,above left,black,font=\stripfontsize] {Microbe \rotatebox[origin=c]{-90}{\ding{49}}};
    \node[text width=(\cardwidth-\strippadding-\stripwidth-2*\textpadding)*1cm,below right,inner sep=0] at (\strippadding+\stripwidth+\textpadding,\cardheight-\textpadding) 
    {   {\captionfontsize \textbf{Opportunistic}}\\ 
        {\textfontsize \textit{Bifidobacterium longum}}\\
        \tikz{\fill (0,0) rectangle (\cardwidth-\strippadding-\stripwidth-2*\textpadding,\ruleheight);}\\
        {\small Synthesizes vitamin B1 (thiamine)}\\
        {\small \small Not resistant
}
        \rule{4cm}{0.4pt}
        {\small \small \textit{Humans are unable to synthesize this vitamin alone}}\\
        
    };
\end{tikzpicture}

&

\begin{tikzpicture}

	\draw[red,fill=red] (5.2,.9) circle (4ex);
	\draw[red,fill=green] (3.7,.9) circle (4ex); 
	\node at (3.7,0.9) {\LARGE \bfseries 2}; 
	\node at (5.2,0.9) {\LARGE \bfseries -1};
    \draw[rounded corners=\cardroundingradius] (0,0) rectangle (\cardwidth,\cardheight);
    \fill[orange,rounded corners=\striproundingradius] (\strippadding,\strippadding) rectangle (\strippadding+\stripwidth,\cardheight-\strippadding) node[rotate=90,above left,black,font=\stripfontsize] {Microbe \rotatebox[origin=c]{-90}{\ding{49}}};
    \node[text width=(\cardwidth-\strippadding-\stripwidth-2*\textpadding)*1cm,below right,inner sep=0] at (\strippadding+\stripwidth+\textpadding,\cardheight-\textpadding) 
    {   {\captionfontsize \textbf{Opportunistic}}\\ 
        {\textfontsize \textit{Bifidobacterium longum}}\\
        \tikz{\fill (0,0) rectangle (\cardwidth-\strippadding-\stripwidth-2*\textpadding,\ruleheight);}\\
        {\small Synthesizes vitamin B1 (thiamine)}\\
        {\small \small Not resistant
}
        \rule{4cm}{0.4pt}
        {\small \small \textit{Humans are unable to synthesize this vitamin alone}}\\
        
    };
\end{tikzpicture}

\\

\begin{tikzpicture}

	\draw[red,fill=red] (5.2,.9) circle (4ex);
	\draw[red,fill=green] (3.7,.9) circle (4ex); 
	\node at (3.7,0.9) {\LARGE \bfseries 2}; 
	\node at (5.2,0.9) {\LARGE \bfseries -1};
    \draw[rounded corners=\cardroundingradius] (0,0) rectangle (\cardwidth,\cardheight);
    \fill[orange,rounded corners=\striproundingradius] (\strippadding,\strippadding) rectangle (\strippadding+\stripwidth,\cardheight-\strippadding) node[rotate=90,above left,black,font=\stripfontsize] {Microbe \rotatebox[origin=c]{-90}{\ding{49}}};
    \node[text width=(\cardwidth-\strippadding-\stripwidth-2*\textpadding)*1cm,below right,inner sep=0] at (\strippadding+\stripwidth+\textpadding,\cardheight-\textpadding) 
    {   {\captionfontsize \textbf{Opportunistic}}\\ 
        {\textfontsize \textit{Escherichia coli}}\\
        \tikz{\fill (0,0) rectangle (\cardwidth-\strippadding-\stripwidth-2*\textpadding,\ruleheight);}\\
        {\small Synthesizes vitamin K}\\
        {\small \small Not resistant
}
        \rule{4cm}{0.4pt}
        {\small \small \textit{E. coli is normally an important part of your gut microbiome}}\\
        
    };
\end{tikzpicture}

&

\begin{tikzpicture}

	\draw[red,fill=red] (5.2,.9) circle (4ex);
	\draw[red,fill=green] (3.7,.9) circle (4ex); 
	\node at (3.7,0.9) {\LARGE \bfseries 2}; 
	\node at (5.2,0.9) {\LARGE \bfseries -1};
    \draw[rounded corners=\cardroundingradius] (0,0) rectangle (\cardwidth,\cardheight);
    \fill[orange,rounded corners=\striproundingradius] (\strippadding,\strippadding) rectangle (\strippadding+\stripwidth,\cardheight-\strippadding) node[rotate=90,above left,black,font=\stripfontsize] {Microbe \rotatebox[origin=c]{-90}{\ding{49}}};
    \node[text width=(\cardwidth-\strippadding-\stripwidth-2*\textpadding)*1cm,below right,inner sep=0] at (\strippadding+\stripwidth+\textpadding,\cardheight-\textpadding) 
    {   {\captionfontsize \textbf{Opportunistic}}\\ 
        {\textfontsize \textit{Escherichia coli}}\\
        \tikz{\fill (0,0) rectangle (\cardwidth-\strippadding-\stripwidth-2*\textpadding,\ruleheight);}\\
        {\small Synthesizes vitamin K}\\
        {\small \small Not resistant
}
        \rule{4cm}{0.4pt}
        {\small \small \textit{E. coli is normally an important part of your gut microbiome}}\\
        
    };
\end{tikzpicture}

&

\begin{tikzpicture}

	\draw[red,fill=red] (5.2,.9) circle (4ex);
	\draw[red,fill=green] (3.7,.9) circle (4ex); 
	\node at (3.7,0.9) {\LARGE \bfseries 1}; 
	\node at (5.2,0.9) {\LARGE \bfseries -1};
    \draw[rounded corners=\cardroundingradius] (0,0) rectangle (\cardwidth,\cardheight);
    \fill[orange,rounded corners=\striproundingradius] (\strippadding,\strippadding) rectangle (\strippadding+\stripwidth,\cardheight-\strippadding) node[rotate=90,above left,black,font=\stripfontsize] {Microbe \rotatebox[origin=c]{-90}{\ding{49}}};
    \node[text width=(\cardwidth-\strippadding-\stripwidth-2*\textpadding)*1cm,below right,inner sep=0] at (\strippadding+\stripwidth+\textpadding,\cardheight-\textpadding) 
    {   {\captionfontsize \textbf{Opportunistic}}\\ 
        {\textfontsize \textit{Bug Op 4}}\\
        \tikz{\fill (0,0) rectangle (\cardwidth-\strippadding-\stripwidth-2*\textpadding,\ruleheight);}\\
        {\small If you have less than 3 microbes in your beneficial zone at end of turn, this becomes a pathogen immediately.  Return to beneficial area once you have 3 microbes there. }\\
        {\small \small Not resistant
}
        \rule{4cm}{0.4pt}
        {\small \small \textit{Some bacteria are like teenagers, they go out of control without supervison}}\\
        
    };
\end{tikzpicture}

\\

\begin{tikzpicture}

	\draw[red,fill=red] (5.2,.9) circle (4ex);
	\draw[red,fill=green] (3.7,.9) circle (4ex); 
	\node at (3.7,0.9) {\LARGE \bfseries 1}; 
	\node at (5.2,0.9) {\LARGE \bfseries -1};
    \draw[rounded corners=\cardroundingradius] (0,0) rectangle (\cardwidth,\cardheight);
    \fill[orange,rounded corners=\striproundingradius] (\strippadding,\strippadding) rectangle (\strippadding+\stripwidth,\cardheight-\strippadding) node[rotate=90,above left,black,font=\stripfontsize] {Microbe \rotatebox[origin=c]{-90}{\ding{49}}};
    \node[text width=(\cardwidth-\strippadding-\stripwidth-2*\textpadding)*1cm,below right,inner sep=0] at (\strippadding+\stripwidth+\textpadding,\cardheight-\textpadding) 
    {   {\captionfontsize \textbf{Opportunistic}}\\ 
        {\textfontsize \textit{Bug Op 4}}\\
        \tikz{\fill (0,0) rectangle (\cardwidth-\strippadding-\stripwidth-2*\textpadding,\ruleheight);}\\
        {\small If you have less than 3 microbes in your beneficial zone at end of turn, this becomes a pathogen immediately.  Return to beneficial area once you have 3 microbes there. }\\
        {\small \small Not resistant
}
        \rule{4cm}{0.4pt}
        {\small \small \textit{Some bacteria are like teenagers, they go out of control without supervison}}\\
        
    };
\end{tikzpicture}

&

\begin{tikzpicture}

	\draw[red,fill=red] (5.2,.9) circle (4ex);
	\draw[red,fill=green] (3.7,.9) circle (4ex); 
	\node at (3.7,0.9) {\LARGE \bfseries 1}; 
	\node at (5.2,0.9) {\LARGE \bfseries -1};
    \draw[rounded corners=\cardroundingradius] (0,0) rectangle (\cardwidth,\cardheight);
    \fill[orange,rounded corners=\striproundingradius] (\strippadding,\strippadding) rectangle (\strippadding+\stripwidth,\cardheight-\strippadding) node[rotate=90,above left,black,font=\stripfontsize] {Microbe \rotatebox[origin=c]{-90}{\ding{49}}};
    \node[text width=(\cardwidth-\strippadding-\stripwidth-2*\textpadding)*1cm,below right,inner sep=0] at (\strippadding+\stripwidth+\textpadding,\cardheight-\textpadding) 
    {   {\captionfontsize \textbf{Opportunistic}}\\ 
        {\textfontsize \textit{Bug Op 4}}\\
        \tikz{\fill (0,0) rectangle (\cardwidth-\strippadding-\stripwidth-2*\textpadding,\ruleheight);}\\
        {\small If you have less than 3 microbes in your beneficial zone at end of turn, this becomes a pathogen immediately.  Return to beneficial area once you have 3 microbes there. }\\
        {\small \small Not resistant
}
        \rule{4cm}{0.4pt}
        {\small \small \textit{Some bacteria are like teenagers, they go out of control without supervison}}\\
        
    };
\end{tikzpicture}

&

\begin{tikzpicture}

	\draw[red,fill=red] (5.2,.9) circle (4ex);
	\draw[red,fill=green] (3.7,.9) circle (4ex); 
	\node at (3.7,0.9) {\LARGE \bfseries 1}; 
	\node at (5.2,0.9) {\LARGE \bfseries -1};
    \draw[rounded corners=\cardroundingradius] (0,0) rectangle (\cardwidth,\cardheight);
    \fill[orange,rounded corners=\striproundingradius] (\strippadding,\strippadding) rectangle (\strippadding+\stripwidth,\cardheight-\strippadding) node[rotate=90,above left,black,font=\stripfontsize] {Microbe \rotatebox[origin=c]{-90}{\ding{49}}};
    \node[text width=(\cardwidth-\strippadding-\stripwidth-2*\textpadding)*1cm,below right,inner sep=0] at (\strippadding+\stripwidth+\textpadding,\cardheight-\textpadding) 
    {   {\captionfontsize \textbf{Opportunistic}}\\ 
        {\textfontsize \textit{Bug Op 4}}\\
        \tikz{\fill (0,0) rectangle (\cardwidth-\strippadding-\stripwidth-2*\textpadding,\ruleheight);}\\
        {\small If you have less than 3 microbes in your beneficial zone at end of turn, this becomes a pathogen immediately.  Return to beneficial area once you have 3 microbes there. }\\
        {\small \small Not resistant
}
        \rule{4cm}{0.4pt}
        {\small \small \textit{Some bacteria are like teenagers, they go out of control without supervison}}\\
        
    };
\end{tikzpicture}

\end{tabular}
\cleardoublepage\begin{tabular}{c c c}

\begin{tikzpicture}

	\draw[red,fill=red] (5.2,.9) circle (4ex);
	\draw[red,fill=green] (3.7,.9) circle (4ex); 
	\node at (3.7,0.9) {\LARGE \bfseries 1}; 
	\node at (5.2,0.9) {\LARGE \bfseries -1};
    \draw[rounded corners=\cardroundingradius] (0,0) rectangle (\cardwidth,\cardheight);
    \fill[orange,rounded corners=\striproundingradius] (\strippadding,\strippadding) rectangle (\strippadding+\stripwidth,\cardheight-\strippadding) node[rotate=90,above left,black,font=\stripfontsize] {Microbe \rotatebox[origin=c]{-90}{\ding{49}}};
    \node[text width=(\cardwidth-\strippadding-\stripwidth-2*\textpadding)*1cm,below right,inner sep=0] at (\strippadding+\stripwidth+\textpadding,\cardheight-\textpadding) 
    {   {\captionfontsize \textbf{Opportunistic}}\\ 
        {\textfontsize \textit{Bug Op 5}}\\
        \tikz{\fill (0,0) rectangle (\cardwidth-\strippadding-\stripwidth-2*\textpadding,\ruleheight);}\\
        {\small (text here)}\\
        {\small \small Not resistant
}
        \rule{4cm}{0.4pt}
        {\small \small \textit{Just hanging around}}\\
        
    };
\end{tikzpicture}

&

\begin{tikzpicture}

	\draw[red,fill=red] (5.2,.9) circle (4ex);
	\draw[red,fill=green] (3.7,.9) circle (4ex); 
	\node at (3.7,0.9) {\LARGE \bfseries 1}; 
	\node at (5.2,0.9) {\LARGE \bfseries -1};
    \draw[rounded corners=\cardroundingradius] (0,0) rectangle (\cardwidth,\cardheight);
    \fill[orange,rounded corners=\striproundingradius] (\strippadding,\strippadding) rectangle (\strippadding+\stripwidth,\cardheight-\strippadding) node[rotate=90,above left,black,font=\stripfontsize] {Microbe \rotatebox[origin=c]{-90}{\ding{49}}};
    \node[text width=(\cardwidth-\strippadding-\stripwidth-2*\textpadding)*1cm,below right,inner sep=0] at (\strippadding+\stripwidth+\textpadding,\cardheight-\textpadding) 
    {   {\captionfontsize \textbf{Opportunistic}}\\ 
        {\textfontsize \textit{Bug Op 5}}\\
        \tikz{\fill (0,0) rectangle (\cardwidth-\strippadding-\stripwidth-2*\textpadding,\ruleheight);}\\
        {\small (text here)}\\
        {\small \small Not resistant
}
        \rule{4cm}{0.4pt}
        {\small \small \textit{Just hanging around}}\\
        
    };
\end{tikzpicture}

&

\begin{tikzpicture}

	\draw[red,fill=red] (5.2,.9) circle (4ex);
	\draw[red,fill=green] (3.7,.9) circle (4ex); 
	\node at (3.7,0.9) {\LARGE \bfseries 1}; 
	\node at (5.2,0.9) {\LARGE \bfseries -1};
    \draw[rounded corners=\cardroundingradius] (0,0) rectangle (\cardwidth,\cardheight);
    \fill[orange,rounded corners=\striproundingradius] (\strippadding,\strippadding) rectangle (\strippadding+\stripwidth,\cardheight-\strippadding) node[rotate=90,above left,black,font=\stripfontsize] {Microbe \rotatebox[origin=c]{-90}{\ding{49}}};
    \node[text width=(\cardwidth-\strippadding-\stripwidth-2*\textpadding)*1cm,below right,inner sep=0] at (\strippadding+\stripwidth+\textpadding,\cardheight-\textpadding) 
    {   {\captionfontsize \textbf{Opportunistic}}\\ 
        {\textfontsize \textit{Bug Op 5}}\\
        \tikz{\fill (0,0) rectangle (\cardwidth-\strippadding-\stripwidth-2*\textpadding,\ruleheight);}\\
        {\small (text here)}\\
        {\small \small Not resistant
}
        \rule{4cm}{0.4pt}
        {\small \small \textit{Just hanging around}}\\
        
    };
\end{tikzpicture}

\\

\begin{tikzpicture}

	\draw[red,fill=red] (5.2,.9) circle (4ex);
	\draw[red,fill=green] (3.7,.9) circle (4ex); 
	\node at (3.7,0.9) {\LARGE \bfseries 1}; 
	\node at (5.2,0.9) {\LARGE \bfseries 0};
    \draw[rounded corners=\cardroundingradius] (0,0) rectangle (\cardwidth,\cardheight);
    \fill[orange,rounded corners=\striproundingradius] (\strippadding,\strippadding) rectangle (\strippadding+\stripwidth,\cardheight-\strippadding) node[rotate=90,above left,black,font=\stripfontsize] {Microbe \rotatebox[origin=c]{-90}{\ding{49}}};
    \node[text width=(\cardwidth-\strippadding-\stripwidth-2*\textpadding)*1cm,below right,inner sep=0] at (\strippadding+\stripwidth+\textpadding,\cardheight-\textpadding) 
    {   {\captionfontsize \textbf{Beneficial}}\\ 
        {\textfontsize \textit{Lactobacillus acidophilus}}\\
        \tikz{\fill (0,0) rectangle (\cardwidth-\strippadding-\stripwidth-2*\textpadding,\ruleheight);}\\
        {\small When this species is in play you can digest lactose}\\
        {\small \small Not resistant
}
        \rule{4cm}{0.4pt}
        {\small \small \textit{This microbe is common in dairy products and probiotics}}\\
        
    };
\end{tikzpicture}

&

\begin{tikzpicture}

	\draw[red,fill=red] (5.2,.9) circle (4ex);
	\draw[red,fill=green] (3.7,.9) circle (4ex); 
	\node at (3.7,0.9) {\LARGE \bfseries 1}; 
	\node at (5.2,0.9) {\LARGE \bfseries 0};
    \draw[rounded corners=\cardroundingradius] (0,0) rectangle (\cardwidth,\cardheight);
    \fill[orange,rounded corners=\striproundingradius] (\strippadding,\strippadding) rectangle (\strippadding+\stripwidth,\cardheight-\strippadding) node[rotate=90,above left,black,font=\stripfontsize] {Microbe \rotatebox[origin=c]{-90}{\ding{49}}};
    \node[text width=(\cardwidth-\strippadding-\stripwidth-2*\textpadding)*1cm,below right,inner sep=0] at (\strippadding+\stripwidth+\textpadding,\cardheight-\textpadding) 
    {   {\captionfontsize \textbf{Beneficial}}\\ 
        {\textfontsize \textit{Lactobacillus acidophilus}}\\
        \tikz{\fill (0,0) rectangle (\cardwidth-\strippadding-\stripwidth-2*\textpadding,\ruleheight);}\\
        {\small When this species is in play you can digest lactose}\\
        {\small \small Not resistant
}
        \rule{4cm}{0.4pt}
        {\small \small \textit{This microbe is common in dairy products and probiotics}}\\
        
    };
\end{tikzpicture}

&

\begin{tikzpicture}

	\draw[red,fill=red] (5.2,.9) circle (4ex);
	\draw[red,fill=green] (3.7,.9) circle (4ex); 
	\node at (3.7,0.9) {\LARGE \bfseries 1}; 
	\node at (5.2,0.9) {\LARGE \bfseries 0};
    \draw[rounded corners=\cardroundingradius] (0,0) rectangle (\cardwidth,\cardheight);
    \fill[orange,rounded corners=\striproundingradius] (\strippadding,\strippadding) rectangle (\strippadding+\stripwidth,\cardheight-\strippadding) node[rotate=90,above left,black,font=\stripfontsize] {Microbe \rotatebox[origin=c]{-90}{\ding{49}}};
    \node[text width=(\cardwidth-\strippadding-\stripwidth-2*\textpadding)*1cm,below right,inner sep=0] at (\strippadding+\stripwidth+\textpadding,\cardheight-\textpadding) 
    {   {\captionfontsize \textbf{Beneficial}}\\ 
        {\textfontsize \textit{Bug Good 2}}\\
        \tikz{\fill (0,0) rectangle (\cardwidth-\strippadding-\stripwidth-2*\textpadding,\ruleheight);}\\
        {\small When this species is in play you can digest grains}\\
        {\small \small Not resistant
}
        \rule{4cm}{0.4pt}
        {\small \small \textit{Something interesting here}}\\
        
    };
\end{tikzpicture}

\\

\begin{tikzpicture}

	\draw[red,fill=red] (5.2,.9) circle (4ex);
	\draw[red,fill=green] (3.7,.9) circle (4ex); 
	\node at (3.7,0.9) {\LARGE \bfseries 1}; 
	\node at (5.2,0.9) {\LARGE \bfseries 0};
    \draw[rounded corners=\cardroundingradius] (0,0) rectangle (\cardwidth,\cardheight);
    \fill[orange,rounded corners=\striproundingradius] (\strippadding,\strippadding) rectangle (\strippadding+\stripwidth,\cardheight-\strippadding) node[rotate=90,above left,black,font=\stripfontsize] {Microbe \rotatebox[origin=c]{-90}{\ding{49}}};
    \node[text width=(\cardwidth-\strippadding-\stripwidth-2*\textpadding)*1cm,below right,inner sep=0] at (\strippadding+\stripwidth+\textpadding,\cardheight-\textpadding) 
    {   {\captionfontsize \textbf{Beneficial}}\\ 
        {\textfontsize \textit{Bug Good 2}}\\
        \tikz{\fill (0,0) rectangle (\cardwidth-\strippadding-\stripwidth-2*\textpadding,\ruleheight);}\\
        {\small When this species is in play you can digest grains}\\
        {\small \small Not resistant
}
        \rule{4cm}{0.4pt}
        {\small \small \textit{Something interesting here}}\\
        
    };
\end{tikzpicture}

&

\begin{tikzpicture}

	\draw[red,fill=red] (5.2,.9) circle (4ex);
	\draw[red,fill=green] (3.7,.9) circle (4ex); 
	\node at (3.7,0.9) {\LARGE \bfseries 1}; 
	\node at (5.2,0.9) {\LARGE \bfseries 0};
    \draw[rounded corners=\cardroundingradius] (0,0) rectangle (\cardwidth,\cardheight);
    \fill[orange,rounded corners=\striproundingradius] (\strippadding,\strippadding) rectangle (\strippadding+\stripwidth,\cardheight-\strippadding) node[rotate=90,above left,black,font=\stripfontsize] {Microbe \rotatebox[origin=c]{-90}{\ding{49}}};
    \node[text width=(\cardwidth-\strippadding-\stripwidth-2*\textpadding)*1cm,below right,inner sep=0] at (\strippadding+\stripwidth+\textpadding,\cardheight-\textpadding) 
    {   {\captionfontsize \textbf{Beneficial}}\\ 
        {\textfontsize \textit{Bug Good 3}}\\
        \tikz{\fill (0,0) rectangle (\cardwidth-\strippadding-\stripwidth-2*\textpadding,\ruleheight);}\\
        {\small When this species is in play you can digest meat}\\
        {\small \small Not resistant
}
        \rule{4cm}{0.4pt}
        {\small \small \textit{Something interesting here}}\\
        
    };
\end{tikzpicture}

&

\begin{tikzpicture}

	\draw[red,fill=red] (5.2,.9) circle (4ex);
	\draw[red,fill=green] (3.7,.9) circle (4ex); 
	\node at (3.7,0.9) {\LARGE \bfseries 1}; 
	\node at (5.2,0.9) {\LARGE \bfseries 0};
    \draw[rounded corners=\cardroundingradius] (0,0) rectangle (\cardwidth,\cardheight);
    \fill[orange,rounded corners=\striproundingradius] (\strippadding,\strippadding) rectangle (\strippadding+\stripwidth,\cardheight-\strippadding) node[rotate=90,above left,black,font=\stripfontsize] {Microbe \rotatebox[origin=c]{-90}{\ding{49}}};
    \node[text width=(\cardwidth-\strippadding-\stripwidth-2*\textpadding)*1cm,below right,inner sep=0] at (\strippadding+\stripwidth+\textpadding,\cardheight-\textpadding) 
    {   {\captionfontsize \textbf{Beneficial}}\\ 
        {\textfontsize \textit{Bug Good 3}}\\
        \tikz{\fill (0,0) rectangle (\cardwidth-\strippadding-\stripwidth-2*\textpadding,\ruleheight);}\\
        {\small When this species is in play you can digest meat}\\
        {\small \small Not resistant
}
        \rule{4cm}{0.4pt}
        {\small \small \textit{Something interesting here}}\\
        
    };
\end{tikzpicture}

\end{tabular}
\cleardoublepage\begin{tabular}{c c c}

\begin{tikzpicture}

	\draw[red,fill=red] (5.2,.9) circle (4ex);
	\draw[red,fill=green] (3.7,.9) circle (4ex); 
	\node at (3.7,0.9) {\LARGE \bfseries 1}; 
	\node at (5.2,0.9) {\LARGE \bfseries 0};
    \draw[rounded corners=\cardroundingradius] (0,0) rectangle (\cardwidth,\cardheight);
    \fill[orange,rounded corners=\striproundingradius] (\strippadding,\strippadding) rectangle (\strippadding+\stripwidth,\cardheight-\strippadding) node[rotate=90,above left,black,font=\stripfontsize] {Microbe \rotatebox[origin=c]{-90}{\ding{49}}};
    \node[text width=(\cardwidth-\strippadding-\stripwidth-2*\textpadding)*1cm,below right,inner sep=0] at (\strippadding+\stripwidth+\textpadding,\cardheight-\textpadding) 
    {   {\captionfontsize \textbf{Beneficial}}\\ 
        {\textfontsize \textit{Lactobacillus acidophilus}}\\
        \tikz{\fill (0,0) rectangle (\cardwidth-\strippadding-\stripwidth-2*\textpadding,\ruleheight);}\\
        {\small When this species is in play you can digest lactose}\\
        {\small \small Not resistant
}
        \rule{4cm}{0.4pt}
        {\small \small \textit{This microbe is common in dairy products and probiotics}}\\
        
    };
\end{tikzpicture}

&

\begin{tikzpicture}

	\draw[red,fill=red] (5.2,.9) circle (4ex);
	\draw[red,fill=green] (3.7,.9) circle (4ex); 
	\node at (3.7,0.9) {\LARGE \bfseries 1}; 
	\node at (5.2,0.9) {\LARGE \bfseries 0};
    \draw[rounded corners=\cardroundingradius] (0,0) rectangle (\cardwidth,\cardheight);
    \fill[orange,rounded corners=\striproundingradius] (\strippadding,\strippadding) rectangle (\strippadding+\stripwidth,\cardheight-\strippadding) node[rotate=90,above left,black,font=\stripfontsize] {Microbe \rotatebox[origin=c]{-90}{\ding{49}}};
    \node[text width=(\cardwidth-\strippadding-\stripwidth-2*\textpadding)*1cm,below right,inner sep=0] at (\strippadding+\stripwidth+\textpadding,\cardheight-\textpadding) 
    {   {\captionfontsize \textbf{Beneficial}}\\ 
        {\textfontsize \textit{Lactobacillus acidophilus}}\\
        \tikz{\fill (0,0) rectangle (\cardwidth-\strippadding-\stripwidth-2*\textpadding,\ruleheight);}\\
        {\small When this species is in play you can digest lactose}\\
        {\small \small Not resistant
}
        \rule{4cm}{0.4pt}
        {\small \small \textit{This microbe is common in dairy products and probiotics}}\\
        
    };
\end{tikzpicture}

&

\begin{tikzpicture}

	\draw[red,fill=red] (5.2,.9) circle (4ex);
	\draw[red,fill=green] (3.7,.9) circle (4ex); 
	\node at (3.7,0.9) {\LARGE \bfseries 1}; 
	\node at (5.2,0.9) {\LARGE \bfseries 0};
    \draw[rounded corners=\cardroundingradius] (0,0) rectangle (\cardwidth,\cardheight);
    \fill[orange,rounded corners=\striproundingradius] (\strippadding,\strippadding) rectangle (\strippadding+\stripwidth,\cardheight-\strippadding) node[rotate=90,above left,black,font=\stripfontsize] {Microbe \rotatebox[origin=c]{-90}{\ding{49}}};
    \node[text width=(\cardwidth-\strippadding-\stripwidth-2*\textpadding)*1cm,below right,inner sep=0] at (\strippadding+\stripwidth+\textpadding,\cardheight-\textpadding) 
    {   {\captionfontsize \textbf{Beneficial}}\\ 
        {\textfontsize \textit{Bug Good 2}}\\
        \tikz{\fill (0,0) rectangle (\cardwidth-\strippadding-\stripwidth-2*\textpadding,\ruleheight);}\\
        {\small When this species is in play you can digest grains}\\
        {\small \small Not resistant
}
        \rule{4cm}{0.4pt}
        {\small \small \textit{Something interesting here}}\\
        
    };
\end{tikzpicture}

\\

\begin{tikzpicture}

	\draw[red,fill=red] (5.2,.9) circle (4ex);
	\draw[red,fill=green] (3.7,.9) circle (4ex); 
	\node at (3.7,0.9) {\LARGE \bfseries 1}; 
	\node at (5.2,0.9) {\LARGE \bfseries 0};
    \draw[rounded corners=\cardroundingradius] (0,0) rectangle (\cardwidth,\cardheight);
    \fill[orange,rounded corners=\striproundingradius] (\strippadding,\strippadding) rectangle (\strippadding+\stripwidth,\cardheight-\strippadding) node[rotate=90,above left,black,font=\stripfontsize] {Microbe \rotatebox[origin=c]{-90}{\ding{49}}};
    \node[text width=(\cardwidth-\strippadding-\stripwidth-2*\textpadding)*1cm,below right,inner sep=0] at (\strippadding+\stripwidth+\textpadding,\cardheight-\textpadding) 
    {   {\captionfontsize \textbf{Beneficial}}\\ 
        {\textfontsize \textit{Bug Good 2}}\\
        \tikz{\fill (0,0) rectangle (\cardwidth-\strippadding-\stripwidth-2*\textpadding,\ruleheight);}\\
        {\small When this species is in play you can digest grains}\\
        {\small \small Not resistant
}
        \rule{4cm}{0.4pt}
        {\small \small \textit{Something interesting here}}\\
        
    };
\end{tikzpicture}

&

\begin{tikzpicture}

	\draw[red,fill=red] (5.2,.9) circle (4ex);
	\draw[red,fill=green] (3.7,.9) circle (4ex); 
	\node at (3.7,0.9) {\LARGE \bfseries 1}; 
	\node at (5.2,0.9) {\LARGE \bfseries 0};
    \draw[rounded corners=\cardroundingradius] (0,0) rectangle (\cardwidth,\cardheight);
    \fill[orange,rounded corners=\striproundingradius] (\strippadding,\strippadding) rectangle (\strippadding+\stripwidth,\cardheight-\strippadding) node[rotate=90,above left,black,font=\stripfontsize] {Microbe \rotatebox[origin=c]{-90}{\ding{49}}};
    \node[text width=(\cardwidth-\strippadding-\stripwidth-2*\textpadding)*1cm,below right,inner sep=0] at (\strippadding+\stripwidth+\textpadding,\cardheight-\textpadding) 
    {   {\captionfontsize \textbf{Beneficial}}\\ 
        {\textfontsize \textit{Bug Good 3}}\\
        \tikz{\fill (0,0) rectangle (\cardwidth-\strippadding-\stripwidth-2*\textpadding,\ruleheight);}\\
        {\small When this species is in play you can digest meat}\\
        {\small \small Not resistant
}
        \rule{4cm}{0.4pt}
        {\small \small \textit{Something interesting here}}\\
        
    };
\end{tikzpicture}

&

\begin{tikzpicture}

	\draw[red,fill=red] (5.2,.9) circle (4ex);
	\draw[red,fill=green] (3.7,.9) circle (4ex); 
	\node at (3.7,0.9) {\LARGE \bfseries 1}; 
	\node at (5.2,0.9) {\LARGE \bfseries 0};
    \draw[rounded corners=\cardroundingradius] (0,0) rectangle (\cardwidth,\cardheight);
    \fill[orange,rounded corners=\striproundingradius] (\strippadding,\strippadding) rectangle (\strippadding+\stripwidth,\cardheight-\strippadding) node[rotate=90,above left,black,font=\stripfontsize] {Microbe \rotatebox[origin=c]{-90}{\ding{49}}};
    \node[text width=(\cardwidth-\strippadding-\stripwidth-2*\textpadding)*1cm,below right,inner sep=0] at (\strippadding+\stripwidth+\textpadding,\cardheight-\textpadding) 
    {   {\captionfontsize \textbf{Beneficial}}\\ 
        {\textfontsize \textit{Bug Good 3}}\\
        \tikz{\fill (0,0) rectangle (\cardwidth-\strippadding-\stripwidth-2*\textpadding,\ruleheight);}\\
        {\small When this species is in play you can digest meat}\\
        {\small \small Not resistant
}
        \rule{4cm}{0.4pt}
        {\small \small \textit{Something interesting here}}\\
        
    };
\end{tikzpicture}

\\

\begin{tikzpicture}

	\draw[red,fill=red] (5.2,.9) circle (4ex);
	\draw[red,fill=green] (3.7,.9) circle (4ex); 
	\node at (3.7,0.9) {\LARGE \bfseries 1}; 
	\node at (5.2,0.9) {\LARGE \bfseries 0};
    \draw[rounded corners=\cardroundingradius] (0,0) rectangle (\cardwidth,\cardheight);
    \fill[pink,rounded corners=\striproundingradius] (\strippadding,\strippadding) rectangle (\strippadding+\stripwidth,\cardheight-\strippadding) node[rotate=90,above left,black,font=\stripfontsize] {Event \rotatebox[origin=c]{-90}{\ding{49}}};
    \node[text width=(\cardwidth-\strippadding-\stripwidth-2*\textpadding)*1cm,below right,inner sep=0] at (\strippadding+\stripwidth+\textpadding,\cardheight-\textpadding) 
    {   {\captionfontsize \textbf{}}\\ 
        {\textfontsize \textit{Prebiotics}}\\
        \tikz{\fill (0,0) rectangle (\cardwidth-\strippadding-\stripwidth-2*\textpadding,\ruleheight);}\\
        {\small This card allows you to play an additional microbe this turn}\\
        {\small \small }
        \rule{4cm}{0.4pt}
        {\small \small \textit{Prebiotics are non-digestable compounds that stimulate bacterial growth or activity}}\\
        
    };
\end{tikzpicture}

&

\begin{tikzpicture}

	\draw[red,fill=red] (5.2,.9) circle (4ex);
	\draw[red,fill=green] (3.7,.9) circle (4ex); 
	\node at (3.7,0.9) {\LARGE \bfseries 1}; 
	\node at (5.2,0.9) {\LARGE \bfseries 0};
    \draw[rounded corners=\cardroundingradius] (0,0) rectangle (\cardwidth,\cardheight);
    \fill[pink,rounded corners=\striproundingradius] (\strippadding,\strippadding) rectangle (\strippadding+\stripwidth,\cardheight-\strippadding) node[rotate=90,above left,black,font=\stripfontsize] {Event \rotatebox[origin=c]{-90}{\ding{49}}};
    \node[text width=(\cardwidth-\strippadding-\stripwidth-2*\textpadding)*1cm,below right,inner sep=0] at (\strippadding+\stripwidth+\textpadding,\cardheight-\textpadding) 
    {   {\captionfontsize \textbf{}}\\ 
        {\textfontsize \textit{Prebiotics}}\\
        \tikz{\fill (0,0) rectangle (\cardwidth-\strippadding-\stripwidth-2*\textpadding,\ruleheight);}\\
        {\small This card allows you to play an additional microbe this turn}\\
        {\small \small }
        \rule{4cm}{0.4pt}
        {\small \small \textit{Prebiotics are non-digestable compounds that stimulate bacterial growth or activity}}\\
        
    };
\end{tikzpicture}

&

\begin{tikzpicture}

	\draw[red,fill=red] (5.2,.9) circle (4ex);
	\draw[red,fill=green] (3.7,.9) circle (4ex); 
	\node at (3.7,0.9) {\LARGE \bfseries 1}; 
	\node at (5.2,0.9) {\LARGE \bfseries 0};
    \draw[rounded corners=\cardroundingradius] (0,0) rectangle (\cardwidth,\cardheight);
    \fill[pink,rounded corners=\striproundingradius] (\strippadding,\strippadding) rectangle (\strippadding+\stripwidth,\cardheight-\strippadding) node[rotate=90,above left,black,font=\stripfontsize] {Event \rotatebox[origin=c]{-90}{\ding{49}}};
    \node[text width=(\cardwidth-\strippadding-\stripwidth-2*\textpadding)*1cm,below right,inner sep=0] at (\strippadding+\stripwidth+\textpadding,\cardheight-\textpadding) 
    {   {\captionfontsize \textbf{}}\\ 
        {\textfontsize \textit{Prebiotics}}\\
        \tikz{\fill (0,0) rectangle (\cardwidth-\strippadding-\stripwidth-2*\textpadding,\ruleheight);}\\
        {\small This card allows you to play an additional microbe this turn}\\
        {\small \small }
        \rule{4cm}{0.4pt}
        {\small \small \textit{Prebiotics are non-digestable compounds that stimulate bacterial growth or activity}}\\
        
    };
\end{tikzpicture}

\end{tabular}
\cleardoublepage\begin{tabular}{c c c}

\begin{tikzpicture}

	\draw[red,fill=red] (5.2,.9) circle (4ex);
	\draw[red,fill=green] (3.7,.9) circle (4ex); 
	\node at (3.7,0.9) {\LARGE \bfseries 0}; 
	\node at (5.2,0.9) {\LARGE \bfseries 0};
    \draw[rounded corners=\cardroundingradius] (0,0) rectangle (\cardwidth,\cardheight);
    \fill[pink,rounded corners=\striproundingradius] (\strippadding,\strippadding) rectangle (\strippadding+\stripwidth,\cardheight-\strippadding) node[rotate=90,above left,black,font=\stripfontsize] {Event \rotatebox[origin=c]{-90}{\ding{49}}};
    \node[text width=(\cardwidth-\strippadding-\stripwidth-2*\textpadding)*1cm,below right,inner sep=0] at (\strippadding+\stripwidth+\textpadding,\cardheight-\textpadding) 
    {   {\captionfontsize \textbf{}}\\ 
        {\textfontsize \textit{Prebiotics}}\\
        \tikz{\fill (0,0) rectangle (\cardwidth-\strippadding-\stripwidth-2*\textpadding,\ruleheight);}\\
        {\small This card allows you to play an additional microbe this turn}\\
        {\small \small }
        \rule{4cm}{0.4pt}
        {\small \small \textit{Prebiotics are non-digestable compounds that stimulate bacterial growth or activity}}\\
        
    };
\end{tikzpicture}

&

\begin{tikzpicture}

	\draw[red,fill=red] (5.2,.9) circle (4ex);
	\draw[red,fill=green] (3.7,.9) circle (4ex); 
	\node at (3.7,0.9) {\LARGE \bfseries 0}; 
	\node at (5.2,0.9) {\LARGE \bfseries 0};
    \draw[rounded corners=\cardroundingradius] (0,0) rectangle (\cardwidth,\cardheight);
    \fill[pink,rounded corners=\striproundingradius] (\strippadding,\strippadding) rectangle (\strippadding+\stripwidth,\cardheight-\strippadding) node[rotate=90,above left,black,font=\stripfontsize] {Event \rotatebox[origin=c]{-90}{\ding{49}}};
    \node[text width=(\cardwidth-\strippadding-\stripwidth-2*\textpadding)*1cm,below right,inner sep=0] at (\strippadding+\stripwidth+\textpadding,\cardheight-\textpadding) 
    {   {\captionfontsize \textbf{}}\\ 
        {\textfontsize \textit{Prebiotics}}\\
        \tikz{\fill (0,0) rectangle (\cardwidth-\strippadding-\stripwidth-2*\textpadding,\ruleheight);}\\
        {\small This card allows you to play an additional microbe this turn}\\
        {\small \small }
        \rule{4cm}{0.4pt}
        {\small \small \textit{Prebiotics are non-digestable compounds that stimulate bacterial growth or activity}}\\
        
    };
\end{tikzpicture}

&

\begin{tikzpicture}

	\draw[red,fill=red] (5.2,.9) circle (4ex);
	\draw[red,fill=green] (3.7,.9) circle (4ex); 
	\node at (3.7,0.9) {\LARGE \bfseries 0}; 
	\node at (5.2,0.9) {\LARGE \bfseries 0};
    \draw[rounded corners=\cardroundingradius] (0,0) rectangle (\cardwidth,\cardheight);
    \fill[pink,rounded corners=\striproundingradius] (\strippadding,\strippadding) rectangle (\strippadding+\stripwidth,\cardheight-\strippadding) node[rotate=90,above left,black,font=\stripfontsize] {Event \rotatebox[origin=c]{-90}{\ding{49}}};
    \node[text width=(\cardwidth-\strippadding-\stripwidth-2*\textpadding)*1cm,below right,inner sep=0] at (\strippadding+\stripwidth+\textpadding,\cardheight-\textpadding) 
    {   {\captionfontsize \textbf{}}\\ 
        {\textfontsize \textit{Prebiotics}}\\
        \tikz{\fill (0,0) rectangle (\cardwidth-\strippadding-\stripwidth-2*\textpadding,\ruleheight);}\\
        {\small This card allows you to play an additional microbe this turn}\\
        {\small \small }
        \rule{4cm}{0.4pt}
        {\small \small \textit{Prebiotics are non-digestable compounds that stimulate bacterial growth or activity}}\\
        
    };
\end{tikzpicture}

\\

\begin{tikzpicture}

	\draw[red,fill=red] (5.2,.9) circle (4ex);
	\draw[red,fill=green] (3.7,.9) circle (4ex); 
	\node at (3.7,0.9) {\LARGE \bfseries 0}; 
	\node at (5.2,0.9) {\LARGE \bfseries -2};
    \draw[rounded corners=\cardroundingradius] (0,0) rectangle (\cardwidth,\cardheight);
    \fill[yellow,rounded corners=\striproundingradius] (\strippadding,\strippadding) rectangle (\strippadding+\stripwidth,\cardheight-\strippadding) node[rotate=90,above left,black,font=\stripfontsize] {Infection \rotatebox[origin=c]{-90}{\ding{49}}};
    \node[text width=(\cardwidth-\strippadding-\stripwidth-2*\textpadding)*1cm,below right,inner sep=0] at (\strippadding+\stripwidth+\textpadding,\cardheight-\textpadding) 
    {   {\captionfontsize \textbf{}}\\ 
        {\textfontsize \textit{Fungal Infection}}\\
        \tikz{\fill (0,0) rectangle (\cardwidth-\strippadding-\stripwidth-2*\textpadding,\ruleheight);}\\
        {\small If target player has less than three microbes in their beneficial zone they lose 2 health at the end of every turn.  Discard when they have three or more microbes in their beneficial zone}\\
        {\small \small }
        \rule{4cm}{0.4pt}
        {\small \small \textit{A healthy microbiom helps protect against fungal infections}}\\
        
    };
\end{tikzpicture}

&

\begin{tikzpicture}

	\draw[red,fill=red] (5.2,.9) circle (4ex);
	\draw[red,fill=green] (3.7,.9) circle (4ex); 
	\node at (3.7,0.9) {\LARGE \bfseries 0}; 
	\node at (5.2,0.9) {\LARGE \bfseries -2};
    \draw[rounded corners=\cardroundingradius] (0,0) rectangle (\cardwidth,\cardheight);
    \fill[yellow,rounded corners=\striproundingradius] (\strippadding,\strippadding) rectangle (\strippadding+\stripwidth,\cardheight-\strippadding) node[rotate=90,above left,black,font=\stripfontsize] {Infection \rotatebox[origin=c]{-90}{\ding{49}}};
    \node[text width=(\cardwidth-\strippadding-\stripwidth-2*\textpadding)*1cm,below right,inner sep=0] at (\strippadding+\stripwidth+\textpadding,\cardheight-\textpadding) 
    {   {\captionfontsize \textbf{}}\\ 
        {\textfontsize \textit{Fungal Infection}}\\
        \tikz{\fill (0,0) rectangle (\cardwidth-\strippadding-\stripwidth-2*\textpadding,\ruleheight);}\\
        {\small If target player has less than three microbes in their beneficial zone they lose 2 health at the end of every turn.  Discard when they have three or more microbes in their beneficial zone}\\
        {\small \small }
        \rule{4cm}{0.4pt}
        {\small \small \textit{A healthy microbiom helps protect against fungal infections}}\\
        
    };
\end{tikzpicture}

&

\begin{tikzpicture}

	\draw[red,fill=red] (5.2,.9) circle (4ex);
	\draw[red,fill=green] (3.7,.9) circle (4ex); 
	\node at (3.7,0.9) {\LARGE \bfseries 2}; 
	\node at (5.2,0.9) {\LARGE \bfseries 0};
    \draw[rounded corners=\cardroundingradius] (0,0) rectangle (\cardwidth,\cardheight);
    \fill[pink,rounded corners=\striproundingradius] (\strippadding,\strippadding) rectangle (\strippadding+\stripwidth,\cardheight-\strippadding) node[rotate=90,above left,black,font=\stripfontsize] {Event \rotatebox[origin=c]{-90}{\ding{49}}};
    \node[text width=(\cardwidth-\strippadding-\stripwidth-2*\textpadding)*1cm,below right,inner sep=0] at (\strippadding+\stripwidth+\textpadding,\cardheight-\textpadding) 
    {   {\captionfontsize \textbf{}}\\ 
        {\textfontsize \textit{Steak}}\\
        \tikz{\fill (0,0) rectangle (\cardwidth-\strippadding-\stripwidth-2*\textpadding,\ruleheight);}\\
        {\small If you have the ability to digest meat, gain 2 health immediately for each microbe with that ability.}\\
        {\small \small }
        \rule{4cm}{0.4pt}
        {\small \small \textit{"Beef� it's what's for dinner"}}\\
        
    };
\end{tikzpicture}

\\

\begin{tikzpicture}

	\draw[red,fill=red] (5.2,.9) circle (4ex);
	\draw[red,fill=green] (3.7,.9) circle (4ex); 
	\node at (3.7,0.9) {\LARGE \bfseries 2}; 
	\node at (5.2,0.9) {\LARGE \bfseries 0};
    \draw[rounded corners=\cardroundingradius] (0,0) rectangle (\cardwidth,\cardheight);
    \fill[pink,rounded corners=\striproundingradius] (\strippadding,\strippadding) rectangle (\strippadding+\stripwidth,\cardheight-\strippadding) node[rotate=90,above left,black,font=\stripfontsize] {Event \rotatebox[origin=c]{-90}{\ding{49}}};
    \node[text width=(\cardwidth-\strippadding-\stripwidth-2*\textpadding)*1cm,below right,inner sep=0] at (\strippadding+\stripwidth+\textpadding,\cardheight-\textpadding) 
    {   {\captionfontsize \textbf{}}\\ 
        {\textfontsize \textit{Steak}}\\
        \tikz{\fill (0,0) rectangle (\cardwidth-\strippadding-\stripwidth-2*\textpadding,\ruleheight);}\\
        {\small If you have the ability to digest meat, gain 2 health immediately for each microbe with that ability.}\\
        {\small \small }
        \rule{4cm}{0.4pt}
        {\small \small \textit{"Beef� it's what's for dinner"}}\\
        
    };
\end{tikzpicture}

&

\begin{tikzpicture}

	\draw[red,fill=red] (5.2,.9) circle (4ex);
	\draw[red,fill=green] (3.7,.9) circle (4ex); 
	\node at (3.7,0.9) {\LARGE \bfseries 2}; 
	\node at (5.2,0.9) {\LARGE \bfseries 0};
    \draw[rounded corners=\cardroundingradius] (0,0) rectangle (\cardwidth,\cardheight);
    \fill[pink,rounded corners=\striproundingradius] (\strippadding,\strippadding) rectangle (\strippadding+\stripwidth,\cardheight-\strippadding) node[rotate=90,above left,black,font=\stripfontsize] {Event \rotatebox[origin=c]{-90}{\ding{49}}};
    \node[text width=(\cardwidth-\strippadding-\stripwidth-2*\textpadding)*1cm,below right,inner sep=0] at (\strippadding+\stripwidth+\textpadding,\cardheight-\textpadding) 
    {   {\captionfontsize \textbf{}}\\ 
        {\textfontsize \textit{Bread}}\\
        \tikz{\fill (0,0) rectangle (\cardwidth-\strippadding-\stripwidth-2*\textpadding,\ruleheight);}\\
        {\small If you have the ability to digest grains, gain 2 health immediately for each microbe with that ability.}\\
        {\small \small }
        \rule{4cm}{0.4pt}
        {\small \small \textit{Not Wonder Bread}}\\
        
    };
\end{tikzpicture}

&

\begin{tikzpicture}

	\draw[red,fill=red] (5.2,.9) circle (4ex);
	\draw[red,fill=green] (3.7,.9) circle (4ex); 
	\node at (3.7,0.9) {\LARGE \bfseries 2}; 
	\node at (5.2,0.9) {\LARGE \bfseries 0};
    \draw[rounded corners=\cardroundingradius] (0,0) rectangle (\cardwidth,\cardheight);
    \fill[pink,rounded corners=\striproundingradius] (\strippadding,\strippadding) rectangle (\strippadding+\stripwidth,\cardheight-\strippadding) node[rotate=90,above left,black,font=\stripfontsize] {Event \rotatebox[origin=c]{-90}{\ding{49}}};
    \node[text width=(\cardwidth-\strippadding-\stripwidth-2*\textpadding)*1cm,below right,inner sep=0] at (\strippadding+\stripwidth+\textpadding,\cardheight-\textpadding) 
    {   {\captionfontsize \textbf{}}\\ 
        {\textfontsize \textit{Bread}}\\
        \tikz{\fill (0,0) rectangle (\cardwidth-\strippadding-\stripwidth-2*\textpadding,\ruleheight);}\\
        {\small If you have the ability to digest grains, gain 2 health immediately for each microbe with that ability.}\\
        {\small \small }
        \rule{4cm}{0.4pt}
        {\small \small \textit{Not Wonder Bread}}\\
        
    };
\end{tikzpicture}

\end{tabular}
\cleardoublepage\begin{tabular}{c c c}

\begin{tikzpicture}

	\draw[red,fill=red] (5.2,.9) circle (4ex);
	\draw[red,fill=green] (3.7,.9) circle (4ex); 
	\node at (3.7,0.9) {\LARGE \bfseries 2}; 
	\node at (5.2,0.9) {\LARGE \bfseries 0};
    \draw[rounded corners=\cardroundingradius] (0,0) rectangle (\cardwidth,\cardheight);
    \fill[pink,rounded corners=\striproundingradius] (\strippadding,\strippadding) rectangle (\strippadding+\stripwidth,\cardheight-\strippadding) node[rotate=90,above left,black,font=\stripfontsize] {Event \rotatebox[origin=c]{-90}{\ding{49}}};
    \node[text width=(\cardwidth-\strippadding-\stripwidth-2*\textpadding)*1cm,below right,inner sep=0] at (\strippadding+\stripwidth+\textpadding,\cardheight-\textpadding) 
    {   {\captionfontsize \textbf{}}\\ 
        {\textfontsize \textit{Milk}}\\
        \tikz{\fill (0,0) rectangle (\cardwidth-\strippadding-\stripwidth-2*\textpadding,\ruleheight);}\\
        {\small If you have the ability to digest lactose, gain 2 health immediately for each microbe with that ability.}\\
        {\small \small }
        \rule{4cm}{0.4pt}
        {\small \small \textit{"Milk� it does a body good"}}\\
        
    };
\end{tikzpicture}

&

\begin{tikzpicture}

	\draw[red,fill=red] (5.2,.9) circle (4ex);
	\draw[red,fill=green] (3.7,.9) circle (4ex); 
	\node at (3.7,0.9) {\LARGE \bfseries 2}; 
	\node at (5.2,0.9) {\LARGE \bfseries 0};
    \draw[rounded corners=\cardroundingradius] (0,0) rectangle (\cardwidth,\cardheight);
    \fill[pink,rounded corners=\striproundingradius] (\strippadding,\strippadding) rectangle (\strippadding+\stripwidth,\cardheight-\strippadding) node[rotate=90,above left,black,font=\stripfontsize] {Event \rotatebox[origin=c]{-90}{\ding{49}}};
    \node[text width=(\cardwidth-\strippadding-\stripwidth-2*\textpadding)*1cm,below right,inner sep=0] at (\strippadding+\stripwidth+\textpadding,\cardheight-\textpadding) 
    {   {\captionfontsize \textbf{}}\\ 
        {\textfontsize \textit{Milk}}\\
        \tikz{\fill (0,0) rectangle (\cardwidth-\strippadding-\stripwidth-2*\textpadding,\ruleheight);}\\
        {\small If you have the ability to digest lactose, gain 2 health immediately for each microbe with that ability.}\\
        {\small \small }
        \rule{4cm}{0.4pt}
        {\small \small \textit{"Milk� it does a body good"}}\\
        
    };
\end{tikzpicture}

&

\begin{tikzpicture}

	\draw[red,fill=red] (5.2,.9) circle (4ex);
	\draw[red,fill=green] (3.7,.9) circle (4ex); 
	\node at (3.7,0.9) {\LARGE \bfseries 0}; 
	\node at (5.2,0.9) {\LARGE \bfseries -4};
    \draw[rounded corners=\cardroundingradius] (0,0) rectangle (\cardwidth,\cardheight);
    \fill[pink,rounded corners=\striproundingradius] (\strippadding,\strippadding) rectangle (\strippadding+\stripwidth,\cardheight-\strippadding) node[rotate=90,above left,black,font=\stripfontsize] {Event \rotatebox[origin=c]{-90}{\ding{49}}};
    \node[text width=(\cardwidth-\strippadding-\stripwidth-2*\textpadding)*1cm,below right,inner sep=0] at (\strippadding+\stripwidth+\textpadding,\cardheight-\textpadding) 
    {   {\captionfontsize \textbf{}}\\ 
        {\textfontsize \textit{Fecal Transplant}}\\
        \tikz{\fill (0,0) rectangle (\cardwidth-\strippadding-\stripwidth-2*\textpadding,\ruleheight);}\\
        {\small This card removes all cards from your pathogen zone (regardless of resistance), you lose 4 health}\\
        {\small \small }
        \rule{4cm}{0.4pt}
        {\small \small \textit{Seriously, these exist}}\\
        
    };
\end{tikzpicture}

\\

\begin{tikzpicture}

	\draw[red,fill=red] (5.2,.9) circle (4ex);
	\draw[red,fill=green] (3.7,.9) circle (4ex); 
	\node at (3.7,0.9) {\LARGE \bfseries 0}; 
	\node at (5.2,0.9) {\LARGE \bfseries -4};
    \draw[rounded corners=\cardroundingradius] (0,0) rectangle (\cardwidth,\cardheight);
    \fill[pink,rounded corners=\striproundingradius] (\strippadding,\strippadding) rectangle (\strippadding+\stripwidth,\cardheight-\strippadding) node[rotate=90,above left,black,font=\stripfontsize] {Event \rotatebox[origin=c]{-90}{\ding{49}}};
    \node[text width=(\cardwidth-\strippadding-\stripwidth-2*\textpadding)*1cm,below right,inner sep=0] at (\strippadding+\stripwidth+\textpadding,\cardheight-\textpadding) 
    {   {\captionfontsize \textbf{}}\\ 
        {\textfontsize \textit{Fecal Transplant}}\\
        \tikz{\fill (0,0) rectangle (\cardwidth-\strippadding-\stripwidth-2*\textpadding,\ruleheight);}\\
        {\small This card removes all cards from your pathogen zone (regardless of resistance), you lose 4 health}\\
        {\small \small }
        \rule{4cm}{0.4pt}
        {\small \small \textit{Seriously, these exist}}\\
        
    };
\end{tikzpicture}

&

\begin{tikzpicture}

	\draw[red,fill=red] (5.2,.9) circle (4ex);
	\draw[red,fill=green] (3.7,.9) circle (4ex); 
	\node at (3.7,0.9) {\LARGE \bfseries 2}; 
	\node at (5.2,0.9) {\LARGE \bfseries 0};
    \draw[rounded corners=\cardroundingradius] (0,0) rectangle (\cardwidth,\cardheight);
    \fill[pink,rounded corners=\striproundingradius] (\strippadding,\strippadding) rectangle (\strippadding+\stripwidth,\cardheight-\strippadding) node[rotate=90,above left,black,font=\stripfontsize] {Event \rotatebox[origin=c]{-90}{\ding{49}}};
    \node[text width=(\cardwidth-\strippadding-\stripwidth-2*\textpadding)*1cm,below right,inner sep=0] at (\strippadding+\stripwidth+\textpadding,\cardheight-\textpadding) 
    {   {\captionfontsize \textbf{}}\\ 
        {\textfontsize \textit{Vitamins}}\\
        \tikz{\fill (0,0) rectangle (\cardwidth-\strippadding-\stripwidth-2*\textpadding,\ruleheight);}\\
        {\small For each vitamin producing microbe in your beneficial zone, gain 2 health immediately}\\
        {\small \small }
        \rule{4cm}{0.4pt}
        {\small \small \textit{Better in your gut than in a pill}}\\
        
    };
\end{tikzpicture}

&

\begin{tikzpicture}

	\draw[red,fill=red] (5.2,.9) circle (4ex);
	\draw[red,fill=green] (3.7,.9) circle (4ex); 
	\node at (3.7,0.9) {\LARGE \bfseries 2}; 
	\node at (5.2,0.9) {\LARGE \bfseries 0};
    \draw[rounded corners=\cardroundingradius] (0,0) rectangle (\cardwidth,\cardheight);
    \fill[pink,rounded corners=\striproundingradius] (\strippadding,\strippadding) rectangle (\strippadding+\stripwidth,\cardheight-\strippadding) node[rotate=90,above left,black,font=\stripfontsize] {Event \rotatebox[origin=c]{-90}{\ding{49}}};
    \node[text width=(\cardwidth-\strippadding-\stripwidth-2*\textpadding)*1cm,below right,inner sep=0] at (\strippadding+\stripwidth+\textpadding,\cardheight-\textpadding) 
    {   {\captionfontsize \textbf{}}\\ 
        {\textfontsize \textit{Vitamins}}\\
        \tikz{\fill (0,0) rectangle (\cardwidth-\strippadding-\stripwidth-2*\textpadding,\ruleheight);}\\
        {\small For each vitamin producing microbe in your beneficial zone, gain 2 health immediately}\\
        {\small \small }
        \rule{4cm}{0.4pt}
        {\small \small \textit{Better in your gut than in a pill}}\\
        
    };
\end{tikzpicture}

\\

\begin{tikzpicture}

	\draw[red,fill=red] (5.2,.9) circle (4ex);
	\draw[red,fill=green] (3.7,.9) circle (4ex); 
	\node at (3.7,0.9) {\LARGE \bfseries 0}; 
	\node at (5.2,0.9) {\LARGE \bfseries 0};
    \draw[rounded corners=\cardroundingradius] (0,0) rectangle (\cardwidth,\cardheight);
    \fill[pink,rounded corners=\striproundingradius] (\strippadding,\strippadding) rectangle (\strippadding+\stripwidth,\cardheight-\strippadding) node[rotate=90,above left,black,font=\stripfontsize] {Event \rotatebox[origin=c]{-90}{\ding{49}}};
    \node[text width=(\cardwidth-\strippadding-\stripwidth-2*\textpadding)*1cm,below right,inner sep=0] at (\strippadding+\stripwidth+\textpadding,\cardheight-\textpadding) 
    {   {\captionfontsize \textbf{}}\\ 
        {\textfontsize \textit{Homeopathy}}\\
        \tikz{\fill (0,0) rectangle (\cardwidth-\strippadding-\stripwidth-2*\textpadding,\ruleheight);}\\
        {\small Play this card for no effect whatsoever}\\
        {\small \small }
        \rule{4cm}{0.4pt}
        {\small \small \textit{But hey, no side effects}}\\
        
    };
\end{tikzpicture}

&

\begin{tikzpicture}

	\draw[red,fill=red] (5.2,.9) circle (4ex);
	\draw[red,fill=green] (3.7,.9) circle (4ex); 
	\node at (3.7,0.9) {\LARGE \bfseries 0}; 
	\node at (5.2,0.9) {\LARGE \bfseries 0};
    \draw[rounded corners=\cardroundingradius] (0,0) rectangle (\cardwidth,\cardheight);
    \fill[pink,rounded corners=\striproundingradius] (\strippadding,\strippadding) rectangle (\strippadding+\stripwidth,\cardheight-\strippadding) node[rotate=90,above left,black,font=\stripfontsize] {Event \rotatebox[origin=c]{-90}{\ding{49}}};
    \node[text width=(\cardwidth-\strippadding-\stripwidth-2*\textpadding)*1cm,below right,inner sep=0] at (\strippadding+\stripwidth+\textpadding,\cardheight-\textpadding) 
    {   {\captionfontsize \textbf{}}\\ 
        {\textfontsize \textit{Homeopathy}}\\
        \tikz{\fill (0,0) rectangle (\cardwidth-\strippadding-\stripwidth-2*\textpadding,\ruleheight);}\\
        {\small Play this card for no effect whatsoever}\\
        {\small \small }
        \rule{4cm}{0.4pt}
        {\small \small \textit{But hey, no side effects}}\\
        
    };
\end{tikzpicture}

&

\begin{tikzpicture}

	\draw[red,fill=red] (5.2,.9) circle (4ex);
	\draw[red,fill=green] (3.7,.9) circle (4ex); 
	\node at (3.7,0.9) {\LARGE \bfseries 0}; 
	\node at (5.2,0.9) {\LARGE \bfseries 0};
    \draw[rounded corners=\cardroundingradius] (0,0) rectangle (\cardwidth,\cardheight);
    \fill[pink,rounded corners=\striproundingradius] (\strippadding,\strippadding) rectangle (\strippadding+\stripwidth,\cardheight-\strippadding) node[rotate=90,above left,black,font=\stripfontsize] {Event \rotatebox[origin=c]{-90}{\ding{49}}};
    \node[text width=(\cardwidth-\strippadding-\stripwidth-2*\textpadding)*1cm,below right,inner sep=0] at (\strippadding+\stripwidth+\textpadding,\cardheight-\textpadding) 
    {   {\captionfontsize \textbf{}}\\ 
        {\textfontsize \textit{Bacteriophage therapy}}\\
        \tikz{\fill (0,0) rectangle (\cardwidth-\strippadding-\stripwidth-2*\textpadding,\ruleheight);}\\
        {\small Destroy any one microbe in play}\\
        {\small \small }
        \rule{4cm}{0.4pt}
        {\small \small \textit{Something interesting here}}\\
        
    };
\end{tikzpicture}

\end{tabular}
\cleardoublepage\begin{tabular}{c c c}

\begin{tikzpicture}

	\draw[red,fill=red] (5.2,.9) circle (4ex);
	\draw[red,fill=green] (3.7,.9) circle (4ex); 
	\node at (3.7,0.9) {\LARGE \bfseries 0}; 
	\node at (5.2,0.9) {\LARGE \bfseries 0};
    \draw[rounded corners=\cardroundingradius] (0,0) rectangle (\cardwidth,\cardheight);
    \fill[pink,rounded corners=\striproundingradius] (\strippadding,\strippadding) rectangle (\strippadding+\stripwidth,\cardheight-\strippadding) node[rotate=90,above left,black,font=\stripfontsize] {Event \rotatebox[origin=c]{-90}{\ding{49}}};
    \node[text width=(\cardwidth-\strippadding-\stripwidth-2*\textpadding)*1cm,below right,inner sep=0] at (\strippadding+\stripwidth+\textpadding,\cardheight-\textpadding) 
    {   {\captionfontsize \textbf{}}\\ 
        {\textfontsize \textit{Bacteriophage therapy}}\\
        \tikz{\fill (0,0) rectangle (\cardwidth-\strippadding-\stripwidth-2*\textpadding,\ruleheight);}\\
        {\small Destroy any one microbe in play}\\
        {\small \small }
        \rule{4cm}{0.4pt}
        {\small \small \textit{Something interesting here}}\\
        
    };
\end{tikzpicture}

&

\begin{tikzpicture}

	\draw[red,fill=red] (5.2,.9) circle (4ex);
	\draw[red,fill=green] (3.7,.9) circle (4ex); 
	\node at (3.7,0.9) {\LARGE \bfseries 0}; 
	\node at (5.2,0.9) {\LARGE \bfseries 0};
    \draw[rounded corners=\cardroundingradius] (0,0) rectangle (\cardwidth,\cardheight);
    \fill[pink,rounded corners=\striproundingradius] (\strippadding,\strippadding) rectangle (\strippadding+\stripwidth,\cardheight-\strippadding) node[rotate=90,above left,black,font=\stripfontsize] {Event  \rotatebox[origin=c]{-90}{\ding{49}}};
    \node[text width=(\cardwidth-\strippadding-\stripwidth-2*\textpadding)*1cm,below right,inner sep=0] at (\strippadding+\stripwidth+\textpadding,\cardheight-\textpadding) 
    {   {\captionfontsize \textbf{}}\\ 
        {\textfontsize \textit{Lateral gene transfer}}\\
        \tikz{\fill (0,0) rectangle (\cardwidth-\strippadding-\stripwidth-2*\textpadding,\ruleheight);}\\
        {\small Play on any plasmid, move that plasmid to another microbe within the same player}\\
        {\small \small }
        \rule{4cm}{0.4pt}
        {\small \small \textit{Microbes are particularly good at sharing}}\\
        
    };
\end{tikzpicture}

&

\begin{tikzpicture}

	\draw[red,fill=red] (5.2,.9) circle (4ex);
	\draw[red,fill=green] (3.7,.9) circle (4ex); 
	\node at (3.7,0.9) {\LARGE \bfseries 0}; 
	\node at (5.2,0.9) {\LARGE \bfseries 0};
    \draw[rounded corners=\cardroundingradius] (0,0) rectangle (\cardwidth,\cardheight);
    \fill[pink,rounded corners=\striproundingradius] (\strippadding,\strippadding) rectangle (\strippadding+\stripwidth,\cardheight-\strippadding) node[rotate=90,above left,black,font=\stripfontsize] {Event  \rotatebox[origin=c]{-90}{\ding{49}}};
    \node[text width=(\cardwidth-\strippadding-\stripwidth-2*\textpadding)*1cm,below right,inner sep=0] at (\strippadding+\stripwidth+\textpadding,\cardheight-\textpadding) 
    {   {\captionfontsize \textbf{}}\\ 
        {\textfontsize \textit{Lateral gene transfer}}\\
        \tikz{\fill (0,0) rectangle (\cardwidth-\strippadding-\stripwidth-2*\textpadding,\ruleheight);}\\
        {\small Play on any plasmid, move that plasmid to another microbe within the same player}\\
        {\small \small }
        \rule{4cm}{0.4pt}
        {\small \small \textit{Microbes are particularly good at sharing}}\\
        
    };
\end{tikzpicture}

\\

\begin{tikzpicture}

	\draw[red,fill=red] (5.2,.9) circle (4ex);
	\draw[red,fill=green] (3.7,.9) circle (4ex); 
	\node at (3.7,0.9) {\LARGE \bfseries 6}; 
	\node at (5.2,0.9) {\LARGE \bfseries 0};
    \draw[rounded corners=\cardroundingradius] (0,0) rectangle (\cardwidth,\cardheight);
    \fill[pink,rounded corners=\striproundingradius] (\strippadding,\strippadding) rectangle (\strippadding+\stripwidth,\cardheight-\strippadding) node[rotate=90,above left,black,font=\stripfontsize] {Event \rotatebox[origin=c]{-90}{\ding{49}}};
    \node[text width=(\cardwidth-\strippadding-\stripwidth-2*\textpadding)*1cm,below right,inner sep=0] at (\strippadding+\stripwidth+\textpadding,\cardheight-\textpadding) 
    {   {\captionfontsize \textbf{}}\\ 
        {\textfontsize \textit{Shepard's pie}}\\
        \tikz{\fill (0,0) rectangle (\cardwidth-\strippadding-\stripwidth-2*\textpadding,\ruleheight);}\\
        {\small If you have the ability to digest meat, grains, and lactose, gain 6 health immediately}\\
        {\small \small }
        \rule{4cm}{0.4pt}
        {\small \small \textit{Mmmmm�. Shepard's pie.}}\\
        
    };
\end{tikzpicture}

&

\begin{tikzpicture}

	\draw[red,fill=red] (5.2,.9) circle (4ex);
	\draw[red,fill=green] (3.7,.9) circle (4ex); 
	\node at (3.7,0.9) {\LARGE \bfseries 6}; 
	\node at (5.2,0.9) {\LARGE \bfseries 0};
    \draw[rounded corners=\cardroundingradius] (0,0) rectangle (\cardwidth,\cardheight);
    \fill[pink,rounded corners=\striproundingradius] (\strippadding,\strippadding) rectangle (\strippadding+\stripwidth,\cardheight-\strippadding) node[rotate=90,above left,black,font=\stripfontsize] {Event \rotatebox[origin=c]{-90}{\ding{49}}};
    \node[text width=(\cardwidth-\strippadding-\stripwidth-2*\textpadding)*1cm,below right,inner sep=0] at (\strippadding+\stripwidth+\textpadding,\cardheight-\textpadding) 
    {   {\captionfontsize \textbf{}}\\ 
        {\textfontsize \textit{Shepard's pie}}\\
        \tikz{\fill (0,0) rectangle (\cardwidth-\strippadding-\stripwidth-2*\textpadding,\ruleheight);}\\
        {\small If you have the ability to digest meat, grains, and lactose, gain 6 health immediately}\\
        {\small \small }
        \rule{4cm}{0.4pt}
        {\small \small \textit{Mmmmm�. Shepard's pie.}}\\
        
    };
\end{tikzpicture}

&

\begin{tikzpicture}

	\draw[red,fill=red] (5.2,.9) circle (4ex);
	\draw[red,fill=green] (3.7,.9) circle (4ex); 
	\node at (3.7,0.9) {\LARGE \bfseries 0}; 
	\node at (5.2,0.9) {\LARGE \bfseries 0};
    \draw[rounded corners=\cardroundingradius] (0,0) rectangle (\cardwidth,\cardheight);
    \fill[pink,rounded corners=\striproundingradius] (\strippadding,\strippadding) rectangle (\strippadding+\stripwidth,\cardheight-\strippadding) node[rotate=90,above left,black,font=\stripfontsize] {Event \rotatebox[origin=c]{-90}{\ding{49}}};
    \node[text width=(\cardwidth-\strippadding-\stripwidth-2*\textpadding)*1cm,below right,inner sep=0] at (\strippadding+\stripwidth+\textpadding,\cardheight-\textpadding) 
    {   {\captionfontsize \textbf{}}\\ 
        {\textfontsize \textit{Probiotics}}\\
        \tikz{\fill (0,0) rectangle (\cardwidth-\strippadding-\stripwidth-2*\textpadding,\ruleheight);}\\
        {\small Draw cards from the deck and place the first non-pathogen in your beneficial area. Reshuffle afterwards. Does not count as playing a microbe this turn}\\
        {\small \small }
        \rule{4cm}{0.4pt}
        {\small \small \textit{Probiotics are defined as microbes that have a putative health benefit when ingested}}\\
        
    };
\end{tikzpicture}

\\

\begin{tikzpicture}

	\draw[red,fill=red] (5.2,.9) circle (4ex);
	\draw[red,fill=green] (3.7,.9) circle (4ex); 
	\node at (3.7,0.9) {\LARGE \bfseries 0}; 
	\node at (5.2,0.9) {\LARGE \bfseries 0};
    \draw[rounded corners=\cardroundingradius] (0,0) rectangle (\cardwidth,\cardheight);
    \fill[pink,rounded corners=\striproundingradius] (\strippadding,\strippadding) rectangle (\strippadding+\stripwidth,\cardheight-\strippadding) node[rotate=90,above left,black,font=\stripfontsize] {Event \rotatebox[origin=c]{-90}{\ding{49}}};
    \node[text width=(\cardwidth-\strippadding-\stripwidth-2*\textpadding)*1cm,below right,inner sep=0] at (\strippadding+\stripwidth+\textpadding,\cardheight-\textpadding) 
    {   {\captionfontsize \textbf{}}\\ 
        {\textfontsize \textit{Probiotics}}\\
        \tikz{\fill (0,0) rectangle (\cardwidth-\strippadding-\stripwidth-2*\textpadding,\ruleheight);}\\
        {\small Draw cards from the deck and place the first non-pathogen in your beneficial area. Reshuffle afterwards. Does not count as playing a microbe this turn}\\
        {\small \small }
        \rule{4cm}{0.4pt}
        {\small \small \textit{Probiotics are defined as microbes that have a putative health benefit when ingested}}\\
        
    };
\end{tikzpicture}

&

\begin{tikzpicture}

	\draw[red,fill=red] (5.2,.9) circle (4ex);
	\draw[red,fill=green] (3.7,.9) circle (4ex); 
	\node at (3.7,0.9) {\LARGE \bfseries 0}; 
	\node at (5.2,0.9) {\LARGE \bfseries 0};
    \draw[rounded corners=\cardroundingradius] (0,0) rectangle (\cardwidth,\cardheight);
    \fill[magenta,rounded corners=\striproundingradius] (\strippadding,\strippadding) rectangle (\strippadding+\stripwidth,\cardheight-\strippadding) node[rotate=90,above left,black,font=\stripfontsize] {Plasmid \rotatebox[origin=c]{-90}{\ding{49}}};
    \node[text width=(\cardwidth-\strippadding-\stripwidth-2*\textpadding)*1cm,below right,inner sep=0] at (\strippadding+\stripwidth+\textpadding,\cardheight-\textpadding) 
    {   {\captionfontsize \textbf{}}\\ 
        {\textfontsize \textit{Tetracycline resistance plasmid}}\\
        \tikz{\fill (0,0) rectangle (\cardwidth-\strippadding-\stripwidth-2*\textpadding,\ruleheight);}\\
        {\small Give any single microbe resistance to Tetracycline}\\
        {\small \small }
        \rule{4cm}{0.4pt}
        {\small \small \textit{A plasmid is a small circular piece of DNA containing genetic information}}\\
        
    };
\end{tikzpicture}

&

\begin{tikzpicture}

	\draw[red,fill=red] (5.2,.9) circle (4ex);
	\draw[red,fill=green] (3.7,.9) circle (4ex); 
	\node at (3.7,0.9) {\LARGE \bfseries 0}; 
	\node at (5.2,0.9) {\LARGE \bfseries 0};
    \draw[rounded corners=\cardroundingradius] (0,0) rectangle (\cardwidth,\cardheight);
    \fill[magenta,rounded corners=\striproundingradius] (\strippadding,\strippadding) rectangle (\strippadding+\stripwidth,\cardheight-\strippadding) node[rotate=90,above left,black,font=\stripfontsize] {Plasmid \rotatebox[origin=c]{-90}{\ding{49}}};
    \node[text width=(\cardwidth-\strippadding-\stripwidth-2*\textpadding)*1cm,below right,inner sep=0] at (\strippadding+\stripwidth+\textpadding,\cardheight-\textpadding) 
    {   {\captionfontsize \textbf{}}\\ 
        {\textfontsize \textit{Tetracycline resistance plasmid}}\\
        \tikz{\fill (0,0) rectangle (\cardwidth-\strippadding-\stripwidth-2*\textpadding,\ruleheight);}\\
        {\small Give any single microbe resistance to Tetracycline}\\
        {\small \small }
        \rule{4cm}{0.4pt}
        {\small \small \textit{A plasmid is a small circular piece of DNA containing genetic information}}\\
        
    };
\end{tikzpicture}

\end{tabular}
\cleardoublepage\begin{tabular}{c c c}

\begin{tikzpicture}

	\draw[red,fill=red] (5.2,.9) circle (4ex);
	\draw[red,fill=green] (3.7,.9) circle (4ex); 
	\node at (3.7,0.9) {\LARGE \bfseries 0}; 
	\node at (5.2,0.9) {\LARGE \bfseries 0};
    \draw[rounded corners=\cardroundingradius] (0,0) rectangle (\cardwidth,\cardheight);
    \fill[magenta,rounded corners=\striproundingradius] (\strippadding,\strippadding) rectangle (\strippadding+\stripwidth,\cardheight-\strippadding) node[rotate=90,above left,black,font=\stripfontsize] {Plasmid \rotatebox[origin=c]{-90}{\ding{49}}};
    \node[text width=(\cardwidth-\strippadding-\stripwidth-2*\textpadding)*1cm,below right,inner sep=0] at (\strippadding+\stripwidth+\textpadding,\cardheight-\textpadding) 
    {   {\captionfontsize \textbf{}}\\ 
        {\textfontsize \textit{Kanamycin resistance plasmid}}\\
        \tikz{\fill (0,0) rectangle (\cardwidth-\strippadding-\stripwidth-2*\textpadding,\ruleheight);}\\
        {\small Give any single microbe resistance to Kanamycin}\\
        {\small \small }
        \rule{4cm}{0.4pt}
        {\small \small \textit{A plasmid is a small circular piece of DNA containing genetic information}}\\
        
    };
\end{tikzpicture}

&

\begin{tikzpicture}

	\draw[red,fill=red] (5.2,.9) circle (4ex);
	\draw[red,fill=green] (3.7,.9) circle (4ex); 
	\node at (3.7,0.9) {\LARGE \bfseries 0}; 
	\node at (5.2,0.9) {\LARGE \bfseries 0};
    \draw[rounded corners=\cardroundingradius] (0,0) rectangle (\cardwidth,\cardheight);
    \fill[magenta,rounded corners=\striproundingradius] (\strippadding,\strippadding) rectangle (\strippadding+\stripwidth,\cardheight-\strippadding) node[rotate=90,above left,black,font=\stripfontsize] {Plasmid \rotatebox[origin=c]{-90}{\ding{49}}};
    \node[text width=(\cardwidth-\strippadding-\stripwidth-2*\textpadding)*1cm,below right,inner sep=0] at (\strippadding+\stripwidth+\textpadding,\cardheight-\textpadding) 
    {   {\captionfontsize \textbf{}}\\ 
        {\textfontsize \textit{Kanamycin resistance plasmid}}\\
        \tikz{\fill (0,0) rectangle (\cardwidth-\strippadding-\stripwidth-2*\textpadding,\ruleheight);}\\
        {\small Give any single microbe resistance to Kanamycin}\\
        {\small \small }
        \rule{4cm}{0.4pt}
        {\small \small \textit{A plasmid is a small circular piece of DNA containing genetic information}}\\
        
    };
\end{tikzpicture}

&

\begin{tikzpicture}

	\draw[red,fill=red] (5.2,.9) circle (4ex);
	\draw[red,fill=green] (3.7,.9) circle (4ex); 
	\node at (3.7,0.9) {\LARGE \bfseries 0}; 
	\node at (5.2,0.9) {\LARGE \bfseries 0};
    \draw[rounded corners=\cardroundingradius] (0,0) rectangle (\cardwidth,\cardheight);
    \fill[magenta,rounded corners=\striproundingradius] (\strippadding,\strippadding) rectangle (\strippadding+\stripwidth,\cardheight-\strippadding) node[rotate=90,above left,black,font=\stripfontsize] {Plasmid \rotatebox[origin=c]{-90}{\ding{49}}};
    \node[text width=(\cardwidth-\strippadding-\stripwidth-2*\textpadding)*1cm,below right,inner sep=0] at (\strippadding+\stripwidth+\textpadding,\cardheight-\textpadding) 
    {   {\captionfontsize \textbf{}}\\ 
        {\textfontsize \textit{Ampicillin resistance plasmid}}\\
        \tikz{\fill (0,0) rectangle (\cardwidth-\strippadding-\stripwidth-2*\textpadding,\ruleheight);}\\
        {\small Give any single microbe resistance to Ampicillin}\\
        {\small \small }
        \rule{4cm}{0.4pt}
        {\small \small \textit{A plasmid is a small circular piece of DNA containing genetic information}}\\
        
    };
\end{tikzpicture}

\\

\begin{tikzpicture}

	\draw[red,fill=red] (5.2,.9) circle (4ex);
	\draw[red,fill=green] (3.7,.9) circle (4ex); 
	\node at (3.7,0.9) {\LARGE \bfseries 0}; 
	\node at (5.2,0.9) {\LARGE \bfseries 0};
    \draw[rounded corners=\cardroundingradius] (0,0) rectangle (\cardwidth,\cardheight);
    \fill[magenta,rounded corners=\striproundingradius] (\strippadding,\strippadding) rectangle (\strippadding+\stripwidth,\cardheight-\strippadding) node[rotate=90,above left,black,font=\stripfontsize] {Plasmid \rotatebox[origin=c]{-90}{\ding{49}}};
    \node[text width=(\cardwidth-\strippadding-\stripwidth-2*\textpadding)*1cm,below right,inner sep=0] at (\strippadding+\stripwidth+\textpadding,\cardheight-\textpadding) 
    {   {\captionfontsize \textbf{}}\\ 
        {\textfontsize \textit{Ampicillin resistance plasmid}}\\
        \tikz{\fill (0,0) rectangle (\cardwidth-\strippadding-\stripwidth-2*\textpadding,\ruleheight);}\\
        {\small Give any single microbe resistance to Ampicillin}\\
        {\small \small }
        \rule{4cm}{0.4pt}
        {\small \small \textit{A plasmid is a small circular piece of DNA containing genetic information}}\\
        
    };
\end{tikzpicture}

&

\begin{tikzpicture}

	\draw[red,fill=red] (5.2,.9) circle (4ex);
	\draw[red,fill=green] (3.7,.9) circle (4ex); 
	\node at (3.7,0.9) {\LARGE \bfseries 0}; 
	\node at (5.2,0.9) {\LARGE \bfseries 0};
    \draw[rounded corners=\cardroundingradius] (0,0) rectangle (\cardwidth,\cardheight);
    \fill[pink,rounded corners=\striproundingradius] (\strippadding,\strippadding) rectangle (\strippadding+\stripwidth,\cardheight-\strippadding) node[rotate=90,above left,black,font=\stripfontsize] {Event \rotatebox[origin=c]{-90}{\ding{49}}};
    \node[text width=(\cardwidth-\strippadding-\stripwidth-2*\textpadding)*1cm,below right,inner sep=0] at (\strippadding+\stripwidth+\textpadding,\cardheight-\textpadding) 
    {   {\captionfontsize \textbf{}}\\ 
        {\textfontsize \textit{Change in health}}\\
        \tikz{\fill (0,0) rectangle (\cardwidth-\strippadding-\stripwidth-2*\textpadding,\ruleheight);}\\
        {\small Move any opportunistic microbe from beneficial to pathogen, or vice versa }\\
        {\small \small }
        \rule{4cm}{0.4pt}
        {\small \small \textit{Waiting in the wings�}}\\
        
    };
\end{tikzpicture}

&

\begin{tikzpicture}

	\draw[red,fill=red] (5.2,.9) circle (4ex);
	\draw[red,fill=green] (3.7,.9) circle (4ex); 
	\node at (3.7,0.9) {\LARGE \bfseries 0}; 
	\node at (5.2,0.9) {\LARGE \bfseries 0};
    \draw[rounded corners=\cardroundingradius] (0,0) rectangle (\cardwidth,\cardheight);
    \fill[pink,rounded corners=\striproundingradius] (\strippadding,\strippadding) rectangle (\strippadding+\stripwidth,\cardheight-\strippadding) node[rotate=90,above left,black,font=\stripfontsize] {Event \rotatebox[origin=c]{-90}{\ding{49}}};
    \node[text width=(\cardwidth-\strippadding-\stripwidth-2*\textpadding)*1cm,below right,inner sep=0] at (\strippadding+\stripwidth+\textpadding,\cardheight-\textpadding) 
    {   {\captionfontsize \textbf{}}\\ 
        {\textfontsize \textit{Change in health}}\\
        \tikz{\fill (0,0) rectangle (\cardwidth-\strippadding-\stripwidth-2*\textpadding,\ruleheight);}\\
        {\small Move any opportunistic microbe from beneficial to pathogen, or vice versa }\\
        {\small \small }
        \rule{4cm}{0.4pt}
        {\small \small \textit{Waiting in the wings�}}\\
        
    };
\end{tikzpicture}

\\

\begin{tikzpicture}

	\draw[red,fill=red] (5.2,.9) circle (4ex);
	\draw[red,fill=green] (3.7,.9) circle (4ex); 
	\node at (3.7,0.9) {\LARGE \bfseries 0}; 
	\node at (5.2,0.9) {\LARGE \bfseries 0};
    \draw[rounded corners=\cardroundingradius] (0,0) rectangle (\cardwidth,\cardheight);
    \fill[pink,rounded corners=\striproundingradius] (\strippadding,\strippadding) rectangle (\strippadding+\stripwidth,\cardheight-\strippadding) node[rotate=90,above left,black,font=\stripfontsize] {Event \rotatebox[origin=c]{-90}{\ding{49}}};
    \node[text width=(\cardwidth-\strippadding-\stripwidth-2*\textpadding)*1cm,below right,inner sep=0] at (\strippadding+\stripwidth+\textpadding,\cardheight-\textpadding) 
    {   {\captionfontsize \textbf{}}\\ 
        {\textfontsize \textit{Change in health}}\\
        \tikz{\fill (0,0) rectangle (\cardwidth-\strippadding-\stripwidth-2*\textpadding,\ruleheight);}\\
        {\small Move any opportunistic microbe from beneficial to pathogen, or vice versa }\\
        {\small \small }
        \rule{4cm}{0.4pt}
        {\small \small \textit{Waiting in the wings�}}\\
        
    };
\end{tikzpicture}

&

\begin{tikzpicture}

	\draw[red,fill=red] (5.2,.9) circle (4ex);
	\draw[red,fill=green] (3.7,.9) circle (4ex); 
	\node at (3.7,0.9) {\LARGE \bfseries 0}; 
	\node at (5.2,0.9) {\LARGE \bfseries 0};
    \draw[rounded corners=\cardroundingradius] (0,0) rectangle (\cardwidth,\cardheight);
    \fill[pink,rounded corners=\striproundingradius] (\strippadding,\strippadding) rectangle (\strippadding+\stripwidth,\cardheight-\strippadding) node[rotate=90,above left,black,font=\stripfontsize] {Event \rotatebox[origin=c]{-90}{\ding{49}}};
    \node[text width=(\cardwidth-\strippadding-\stripwidth-2*\textpadding)*1cm,below right,inner sep=0] at (\strippadding+\stripwidth+\textpadding,\cardheight-\textpadding) 
    {   {\captionfontsize \textbf{}}\\ 
        {\textfontsize \textit{Change in health}}\\
        \tikz{\fill (0,0) rectangle (\cardwidth-\strippadding-\stripwidth-2*\textpadding,\ruleheight);}\\
        {\small Move any opportunistic microbe from beneficial to pathogen, or vice versa }\\
        {\small \small }
        \rule{4cm}{0.4pt}
        {\small \small \textit{Waiting in the wings�}}\\
        
    };
\end{tikzpicture}

&

\begin{tikzpicture}

	\draw[red,fill=red] (5.2,.9) circle (4ex);
	\draw[red,fill=green] (3.7,.9) circle (4ex); 
	\node at (3.7,0.9) {\LARGE \bfseries 0}; 
	\node at (5.2,0.9) {\LARGE \bfseries 0};
    \draw[rounded corners=\cardroundingradius] (0,0) rectangle (\cardwidth,\cardheight);
    \fill[pink,rounded corners=\striproundingradius] (\strippadding,\strippadding) rectangle (\strippadding+\stripwidth,\cardheight-\strippadding) node[rotate=90,above left,black,font=\stripfontsize] {Event \rotatebox[origin=c]{-90}{\ding{49}}};
    \node[text width=(\cardwidth-\strippadding-\stripwidth-2*\textpadding)*1cm,below right,inner sep=0] at (\strippadding+\stripwidth+\textpadding,\cardheight-\textpadding) 
    {   {\captionfontsize \textbf{}}\\ 
        {\textfontsize \textit{Change in health}}\\
        \tikz{\fill (0,0) rectangle (\cardwidth-\strippadding-\stripwidth-2*\textpadding,\ruleheight);}\\
        {\small Move any opportunistic microbe from beneficial to pathogen, or vice versa }\\
        {\small \small }
        \rule{4cm}{0.4pt}
        {\small \small \textit{Waiting in the wings�}}\\
        
    };
\end{tikzpicture}

\end{tabular}
\cleardoublepage\begin{tabular}{c c c}

\begin{tikzpicture}

	\draw[red,fill=red] (5.2,.9) circle (4ex);
	\draw[red,fill=green] (3.7,.9) circle (4ex); 
	\node at (3.7,0.9) {\LARGE \bfseries 0}; 
	\node at (5.2,0.9) {\LARGE \bfseries 0};
    \draw[rounded corners=\cardroundingradius] (0,0) rectangle (\cardwidth,\cardheight);
    \fill[pink,rounded corners=\striproundingradius] (\strippadding,\strippadding) rectangle (\strippadding+\stripwidth,\cardheight-\strippadding) node[rotate=90,above left,black,font=\stripfontsize] {Event \rotatebox[origin=c]{-90}{\ding{49}}};
    \node[text width=(\cardwidth-\strippadding-\stripwidth-2*\textpadding)*1cm,below right,inner sep=0] at (\strippadding+\stripwidth+\textpadding,\cardheight-\textpadding) 
    {   {\captionfontsize \textbf{}}\\ 
        {\textfontsize \textit{Change in health}}\\
        \tikz{\fill (0,0) rectangle (\cardwidth-\strippadding-\stripwidth-2*\textpadding,\ruleheight);}\\
        {\small Move any opportunistic microbe from beneficial to pathogen, or vice versa }\\
        {\small \small }
        \rule{4cm}{0.4pt}
        {\small \small \textit{Waiting in the wings�}}\\
        
    };
\end{tikzpicture}

&

\begin{tikzpicture}

	\draw[red,fill=red] (5.2,.9) circle (4ex);
	\draw[red,fill=green] (3.7,.9) circle (4ex); 
	\node at (3.7,0.9) {\LARGE \bfseries 0}; 
	\node at (5.2,0.9) {\LARGE \bfseries -2};
    \draw[rounded corners=\cardroundingradius] (0,0) rectangle (\cardwidth,\cardheight);
    \fill[pink,rounded corners=\striproundingradius] (\strippadding,\strippadding) rectangle (\strippadding+\stripwidth,\cardheight-\strippadding) node[rotate=90,above left,black,font=\stripfontsize] {Event \rotatebox[origin=c]{-90}{\ding{49}}};
    \node[text width=(\cardwidth-\strippadding-\stripwidth-2*\textpadding)*1cm,below right,inner sep=0] at (\strippadding+\stripwidth+\textpadding,\cardheight-\textpadding) 
    {   {\captionfontsize \textbf{}}\\ 
        {\textfontsize \textit{Tetracycline}}\\
        \tikz{\fill (0,0) rectangle (\cardwidth-\strippadding-\stripwidth-2*\textpadding,\ruleheight);}\\
        {\small Target player may remove a non-tetracycline resistant pathogen of their choice, and loses half of the non-tetracycline resistant microbes in their beneficial zone (rounded up) and 2 health}\\
        {\small \small }
        \rule{4cm}{0.4pt}
        {\small \small \textit{A once widely-used antibiotic, resistance is now common}}\\
        
    };
\end{tikzpicture}

&

\begin{tikzpicture}

	\draw[red,fill=red] (5.2,.9) circle (4ex);
	\draw[red,fill=green] (3.7,.9) circle (4ex); 
	\node at (3.7,0.9) {\LARGE \bfseries 0}; 
	\node at (5.2,0.9) {\LARGE \bfseries -2};
    \draw[rounded corners=\cardroundingradius] (0,0) rectangle (\cardwidth,\cardheight);
    \fill[pink,rounded corners=\striproundingradius] (\strippadding,\strippadding) rectangle (\strippadding+\stripwidth,\cardheight-\strippadding) node[rotate=90,above left,black,font=\stripfontsize] {Event \rotatebox[origin=c]{-90}{\ding{49}}};
    \node[text width=(\cardwidth-\strippadding-\stripwidth-2*\textpadding)*1cm,below right,inner sep=0] at (\strippadding+\stripwidth+\textpadding,\cardheight-\textpadding) 
    {   {\captionfontsize \textbf{}}\\ 
        {\textfontsize \textit{Tetracycline}}\\
        \tikz{\fill (0,0) rectangle (\cardwidth-\strippadding-\stripwidth-2*\textpadding,\ruleheight);}\\
        {\small Target player may remove a non-tetracycline resistant pathogen of their choice, and loses half of the non-tetracycline resistant microbes in their beneficial zone (rounded up) and 2 health}\\
        {\small \small }
        \rule{4cm}{0.4pt}
        {\small \small \textit{A once widely-used antibiotic, resistance is now common}}\\
        
    };
\end{tikzpicture}

\\

\begin{tikzpicture}

	\draw[red,fill=red] (5.2,.9) circle (4ex);
	\draw[red,fill=green] (3.7,.9) circle (4ex); 
	\node at (3.7,0.9) {\LARGE \bfseries 0}; 
	\node at (5.2,0.9) {\LARGE \bfseries -2};
    \draw[rounded corners=\cardroundingradius] (0,0) rectangle (\cardwidth,\cardheight);
    \fill[pink,rounded corners=\striproundingradius] (\strippadding,\strippadding) rectangle (\strippadding+\stripwidth,\cardheight-\strippadding) node[rotate=90,above left,black,font=\stripfontsize] {Event \rotatebox[origin=c]{-90}{\ding{49}}};
    \node[text width=(\cardwidth-\strippadding-\stripwidth-2*\textpadding)*1cm,below right,inner sep=0] at (\strippadding+\stripwidth+\textpadding,\cardheight-\textpadding) 
    {   {\captionfontsize \textbf{}}\\ 
        {\textfontsize \textit{Tetracycline}}\\
        \tikz{\fill (0,0) rectangle (\cardwidth-\strippadding-\stripwidth-2*\textpadding,\ruleheight);}\\
        {\small Target player may remove a non-tetracycline resistant pathogen of their choice, and loses half of the non-tetracycline resistant microbes in their beneficial zone (rounded up) and 2 health}\\
        {\small \small }
        \rule{4cm}{0.4pt}
        {\small \small \textit{A once widely-used antibiotic, resistance is now common}}\\
        
    };
\end{tikzpicture}

&

\begin{tikzpicture}

	\draw[red,fill=red] (5.2,.9) circle (4ex);
	\draw[red,fill=green] (3.7,.9) circle (4ex); 
	\node at (3.7,0.9) {\LARGE \bfseries 0}; 
	\node at (5.2,0.9) {\LARGE \bfseries -2};
    \draw[rounded corners=\cardroundingradius] (0,0) rectangle (\cardwidth,\cardheight);
    \fill[pink,rounded corners=\striproundingradius] (\strippadding,\strippadding) rectangle (\strippadding+\stripwidth,\cardheight-\strippadding) node[rotate=90,above left,black,font=\stripfontsize] {Event \rotatebox[origin=c]{-90}{\ding{49}}};
    \node[text width=(\cardwidth-\strippadding-\stripwidth-2*\textpadding)*1cm,below right,inner sep=0] at (\strippadding+\stripwidth+\textpadding,\cardheight-\textpadding) 
    {   {\captionfontsize \textbf{}}\\ 
        {\textfontsize \textit{Kanamycin}}\\
        \tikz{\fill (0,0) rectangle (\cardwidth-\strippadding-\stripwidth-2*\textpadding,\ruleheight);}\\
        {\small Target player may remove a non-kanamycin resistant pathogen of their choice, and loses half of the non-kanamycin resistant microbes in their beneficial zone (rounded up) and 2 health}\\
        {\small \small }
        \rule{4cm}{0.4pt}
        {\small \small \textit{A widely-used antibiotic, produced by Streptomyces kanamyceticus}}\\
        
    };
\end{tikzpicture}

&

\begin{tikzpicture}

	\draw[red,fill=red] (5.2,.9) circle (4ex);
	\draw[red,fill=green] (3.7,.9) circle (4ex); 
	\node at (3.7,0.9) {\LARGE \bfseries 0}; 
	\node at (5.2,0.9) {\LARGE \bfseries -2};
    \draw[rounded corners=\cardroundingradius] (0,0) rectangle (\cardwidth,\cardheight);
    \fill[pink,rounded corners=\striproundingradius] (\strippadding,\strippadding) rectangle (\strippadding+\stripwidth,\cardheight-\strippadding) node[rotate=90,above left,black,font=\stripfontsize] {Event \rotatebox[origin=c]{-90}{\ding{49}}};
    \node[text width=(\cardwidth-\strippadding-\stripwidth-2*\textpadding)*1cm,below right,inner sep=0] at (\strippadding+\stripwidth+\textpadding,\cardheight-\textpadding) 
    {   {\captionfontsize \textbf{}}\\ 
        {\textfontsize \textit{Kanamycin}}\\
        \tikz{\fill (0,0) rectangle (\cardwidth-\strippadding-\stripwidth-2*\textpadding,\ruleheight);}\\
        {\small Target player may remove a non-kanamycin resistant pathogen of their choice, and loses half of the non-kanamycin resistant microbes in their beneficial zone (rounded up) and 2 health}\\
        {\small \small }
        \rule{4cm}{0.4pt}
        {\small \small \textit{A widely-used antibiotic, produced by Streptomyces kanamyceticus}}\\
        
    };
\end{tikzpicture}

\\

\begin{tikzpicture}

	\draw[red,fill=red] (5.2,.9) circle (4ex);
	\draw[red,fill=green] (3.7,.9) circle (4ex); 
	\node at (3.7,0.9) {\LARGE \bfseries 0}; 
	\node at (5.2,0.9) {\LARGE \bfseries -2};
    \draw[rounded corners=\cardroundingradius] (0,0) rectangle (\cardwidth,\cardheight);
    \fill[pink,rounded corners=\striproundingradius] (\strippadding,\strippadding) rectangle (\strippadding+\stripwidth,\cardheight-\strippadding) node[rotate=90,above left,black,font=\stripfontsize] {Event \rotatebox[origin=c]{-90}{\ding{49}}};
    \node[text width=(\cardwidth-\strippadding-\stripwidth-2*\textpadding)*1cm,below right,inner sep=0] at (\strippadding+\stripwidth+\textpadding,\cardheight-\textpadding) 
    {   {\captionfontsize \textbf{}}\\ 
        {\textfontsize \textit{Kanamycin}}\\
        \tikz{\fill (0,0) rectangle (\cardwidth-\strippadding-\stripwidth-2*\textpadding,\ruleheight);}\\
        {\small Target player may remove a non-kanamycin resistant pathogen of their choice, and loses half of the non-kanamycin resistant microbes in their beneficial zone (rounded up) and 2 health}\\
        {\small \small }
        \rule{4cm}{0.4pt}
        {\small \small \textit{A widely-used antibiotic, produced by Streptomyces kanamyceticus}}\\
        
    };
\end{tikzpicture}

&

\begin{tikzpicture}

	\draw[red,fill=red] (5.2,.9) circle (4ex);
	\draw[red,fill=green] (3.7,.9) circle (4ex); 
	\node at (3.7,0.9) {\LARGE \bfseries 0}; 
	\node at (5.2,0.9) {\LARGE \bfseries -1};
    \draw[rounded corners=\cardroundingradius] (0,0) rectangle (\cardwidth,\cardheight);
    \fill[pink,rounded corners=\striproundingradius] (\strippadding,\strippadding) rectangle (\strippadding+\stripwidth,\cardheight-\strippadding) node[rotate=90,above left,black,font=\stripfontsize] {Event \rotatebox[origin=c]{-90}{\ding{49}}};
    \node[text width=(\cardwidth-\strippadding-\stripwidth-2*\textpadding)*1cm,below right,inner sep=0] at (\strippadding+\stripwidth+\textpadding,\cardheight-\textpadding) 
    {   {\captionfontsize \textbf{}}\\ 
        {\textfontsize \textit{Ampicillin}}\\
        \tikz{\fill (0,0) rectangle (\cardwidth-\strippadding-\stripwidth-2*\textpadding,\ruleheight);}\\
        {\small Target player may remove a non-ampicillin resistant pathogen of their choice, and loses half of the non-ampicillin resistant microbes in their beneficial zone (rounded up) and 1 health}\\
        {\small \small }
        \rule{4cm}{0.4pt}
        {\small \small \textit{An antibiotic from the penicillin family}}\\
        
    };
\end{tikzpicture}

&

\begin{tikzpicture}

	\draw[red,fill=red] (5.2,.9) circle (4ex);
	\draw[red,fill=green] (3.7,.9) circle (4ex); 
	\node at (3.7,0.9) {\LARGE \bfseries 0}; 
	\node at (5.2,0.9) {\LARGE \bfseries -1};
    \draw[rounded corners=\cardroundingradius] (0,0) rectangle (\cardwidth,\cardheight);
    \fill[pink,rounded corners=\striproundingradius] (\strippadding,\strippadding) rectangle (\strippadding+\stripwidth,\cardheight-\strippadding) node[rotate=90,above left,black,font=\stripfontsize] {Event \rotatebox[origin=c]{-90}{\ding{49}}};
    \node[text width=(\cardwidth-\strippadding-\stripwidth-2*\textpadding)*1cm,below right,inner sep=0] at (\strippadding+\stripwidth+\textpadding,\cardheight-\textpadding) 
    {   {\captionfontsize \textbf{}}\\ 
        {\textfontsize \textit{Ampicillin}}\\
        \tikz{\fill (0,0) rectangle (\cardwidth-\strippadding-\stripwidth-2*\textpadding,\ruleheight);}\\
        {\small Target player may remove a non-ampicillin resistant pathogen of their choice, and loses half of the non-ampicillin resistant microbes in their beneficial zone (rounded up) and 1 health}\\
        {\small \small }
        \rule{4cm}{0.4pt}
        {\small \small \textit{An antibiotic from the penicillin family}}\\
        
    };
\end{tikzpicture}

\end{tabular}
\cleardoublepage\begin{tabular}{c c c}

\begin{tikzpicture}

	\draw[red,fill=red] (5.2,.9) circle (4ex);
	\draw[red,fill=green] (3.7,.9) circle (4ex); 
	\node at (3.7,0.9) {\LARGE \bfseries 0}; 
	\node at (5.2,0.9) {\LARGE \bfseries -1};
    \draw[rounded corners=\cardroundingradius] (0,0) rectangle (\cardwidth,\cardheight);
    \fill[pink,rounded corners=\striproundingradius] (\strippadding,\strippadding) rectangle (\strippadding+\stripwidth,\cardheight-\strippadding) node[rotate=90,above left,black,font=\stripfontsize] {Event \rotatebox[origin=c]{-90}{\ding{49}}};
    \node[text width=(\cardwidth-\strippadding-\stripwidth-2*\textpadding)*1cm,below right,inner sep=0] at (\strippadding+\stripwidth+\textpadding,\cardheight-\textpadding) 
    {   {\captionfontsize \textbf{}}\\ 
        {\textfontsize \textit{Ampicillin}}\\
        \tikz{\fill (0,0) rectangle (\cardwidth-\strippadding-\stripwidth-2*\textpadding,\ruleheight);}\\
        {\small Target player may remove a non-ampicillin resistant pathogen of their choice, and loses half of the non-ampicillin resistant microbes in their beneficial zone (rounded up) and 1 health}\\
        {\small \small }
        \rule{4cm}{0.4pt}
        {\small \small \textit{An antibiotic from the penicillin family}}\\
        
    };
\end{tikzpicture}

&

\begin{tikzpicture}

	\draw[red,fill=red] (5.2,.9) circle (4ex);
	\draw[red,fill=green] (3.7,.9) circle (4ex); 
	\node at (3.7,0.9) {\LARGE \bfseries 0}; 
	\node at (5.2,0.9) {\LARGE \bfseries 0};
    \draw[rounded corners=\cardroundingradius] (0,0) rectangle (\cardwidth,\cardheight);
    \fill[pink,rounded corners=\striproundingradius] (\strippadding,\strippadding) rectangle (\strippadding+\stripwidth,\cardheight-\strippadding) node[rotate=90,above left,black,font=\stripfontsize] {Event \rotatebox[origin=c]{-90}{\ding{49}}};
    \node[text width=(\cardwidth-\strippadding-\stripwidth-2*\textpadding)*1cm,below right,inner sep=0] at (\strippadding+\stripwidth+\textpadding,\cardheight-\textpadding) 
    {   {\captionfontsize \textbf{}}\\ 
        {\textfontsize \textit{Microbial Diversity}}\\
        \tikz{\fill (0,0) rectangle (\cardwidth-\strippadding-\stripwidth-2*\textpadding,\ruleheight);}\\
        {\small If you have at least 4 microbes in your beneficial zone, remove a microbe from your pathogen zone}\\
        {\small \small }
        \rule{4cm}{0.4pt}
        {\small \small \textit{There appears to be a correlation between diversity of microbiota and health}}\\
        
    };
\end{tikzpicture}

&

\begin{tikzpicture}

	\draw[red,fill=red] (5.2,.9) circle (4ex);
	\draw[red,fill=green] (3.7,.9) circle (4ex); 
	\node at (3.7,0.9) {\LARGE \bfseries 0}; 
	\node at (5.2,0.9) {\LARGE \bfseries 0};
    \draw[rounded corners=\cardroundingradius] (0,0) rectangle (\cardwidth,\cardheight);
    \fill[pink,rounded corners=\striproundingradius] (\strippadding,\strippadding) rectangle (\strippadding+\stripwidth,\cardheight-\strippadding) node[rotate=90,above left,black,font=\stripfontsize] {Event \rotatebox[origin=c]{-90}{\ding{49}}};
    \node[text width=(\cardwidth-\strippadding-\stripwidth-2*\textpadding)*1cm,below right,inner sep=0] at (\strippadding+\stripwidth+\textpadding,\cardheight-\textpadding) 
    {   {\captionfontsize \textbf{}}\\ 
        {\textfontsize \textit{Microbial Diversity}}\\
        \tikz{\fill (0,0) rectangle (\cardwidth-\strippadding-\stripwidth-2*\textpadding,\ruleheight);}\\
        {\small If you have at least 4 microbes in your beneficial zone, remove a microbe from your pathogen zone}\\
        {\small \small }
        \rule{4cm}{0.4pt}
        {\small \small \textit{There appears to be a correlation between diversity of microbiota and health}}\\
        
    };
\end{tikzpicture}

\\

\begin{tikzpicture}

	\draw[red,fill=red] (5.2,.9) circle (4ex);
	\draw[red,fill=green] (3.7,.9) circle (4ex); 
	\node at (3.7,0.9) {\LARGE \bfseries 0}; 
	\node at (5.2,0.9) {\LARGE \bfseries -2};
    \draw[rounded corners=\cardroundingradius] (0,0) rectangle (\cardwidth,\cardheight);
    \fill[pink,rounded corners=\striproundingradius] (\strippadding,\strippadding) rectangle (\strippadding+\stripwidth,\cardheight-\strippadding) node[rotate=90,above left,black,font=\stripfontsize] {Event \rotatebox[origin=c]{-90}{\ding{49}}};
    \node[text width=(\cardwidth-\strippadding-\stripwidth-2*\textpadding)*1cm,below right,inner sep=0] at (\strippadding+\stripwidth+\textpadding,\cardheight-\textpadding) 
    {   {\captionfontsize \textbf{}}\\ 
        {\textfontsize \textit{Go to work sick}}\\
        \tikz{\fill (0,0) rectangle (\cardwidth-\strippadding-\stripwidth-2*\textpadding,\ruleheight);}\\
        {\small Lose 2 health and give a microbe from your pathogen zone to target player}\\
        {\small \small }
        \rule{4cm}{0.4pt}
        {\small \small \textit{Stay home!}}\\
        
    };
\end{tikzpicture}

&

\begin{tikzpicture}

	\draw[red,fill=red] (5.2,.9) circle (4ex);
	\draw[red,fill=green] (3.7,.9) circle (4ex); 
	\node at (3.7,0.9) {\LARGE \bfseries 0}; 
	\node at (5.2,0.9) {\LARGE \bfseries -2};
    \draw[rounded corners=\cardroundingradius] (0,0) rectangle (\cardwidth,\cardheight);
    \fill[pink,rounded corners=\striproundingradius] (\strippadding,\strippadding) rectangle (\strippadding+\stripwidth,\cardheight-\strippadding) node[rotate=90,above left,black,font=\stripfontsize] {Event \rotatebox[origin=c]{-90}{\ding{49}}};
    \node[text width=(\cardwidth-\strippadding-\stripwidth-2*\textpadding)*1cm,below right,inner sep=0] at (\strippadding+\stripwidth+\textpadding,\cardheight-\textpadding) 
    {   {\captionfontsize \textbf{}}\\ 
        {\textfontsize \textit{Go to work sick}}\\
        \tikz{\fill (0,0) rectangle (\cardwidth-\strippadding-\stripwidth-2*\textpadding,\ruleheight);}\\
        {\small Lose 2 health and give a microbe from your pathogen zone to target player}\\
        {\small \small }
        \rule{4cm}{0.4pt}
        {\small \small \textit{Stay home!}}\\
        
    };
\end{tikzpicture}

&

\begin{tikzpicture}

	\draw[red,fill=red] (5.2,.9) circle (4ex);
	\draw[red,fill=green] (3.7,.9) circle (4ex); 
	\node at (3.7,0.9) {\LARGE \bfseries 0}; 
	\node at (5.2,0.9) {\LARGE \bfseries 0};
    \draw[rounded corners=\cardroundingradius] (0,0) rectangle (\cardwidth,\cardheight);
    \fill[pink,rounded corners=\striproundingradius] (\strippadding,\strippadding) rectangle (\strippadding+\stripwidth,\cardheight-\strippadding) node[rotate=90,above left,black,font=\stripfontsize] {Event \rotatebox[origin=c]{-90}{\ding{49}}};
    \node[text width=(\cardwidth-\strippadding-\stripwidth-2*\textpadding)*1cm,below right,inner sep=0] at (\strippadding+\stripwidth+\textpadding,\cardheight-\textpadding) 
    {   {\captionfontsize \textbf{}}\\ 
        {\textfontsize \textit{Airplane trip}}\\
        \tikz{\fill (0,0) rectangle (\cardwidth-\strippadding-\stripwidth-2*\textpadding,\ruleheight);}\\
        {\small Each player passes one microbe in play to the player on their left}\\
        {\small \small }
        \rule{4cm}{0.4pt}
        {\small \small \textit{Sharing is caring}}\\
        
    };
\end{tikzpicture}

\\

\begin{tikzpicture}

	\draw[red,fill=red] (5.2,.9) circle (4ex);
	\draw[red,fill=green] (3.7,.9) circle (4ex); 
	\node at (3.7,0.9) {\LARGE \bfseries 0}; 
	\node at (5.2,0.9) {\LARGE \bfseries 0};
    \draw[rounded corners=\cardroundingradius] (0,0) rectangle (\cardwidth,\cardheight);
    \fill[pink,rounded corners=\striproundingradius] (\strippadding,\strippadding) rectangle (\strippadding+\stripwidth,\cardheight-\strippadding) node[rotate=90,above left,black,font=\stripfontsize] {Event \rotatebox[origin=c]{-90}{\ding{49}}};
    \node[text width=(\cardwidth-\strippadding-\stripwidth-2*\textpadding)*1cm,below right,inner sep=0] at (\strippadding+\stripwidth+\textpadding,\cardheight-\textpadding) 
    {   {\captionfontsize \textbf{}}\\ 
        {\textfontsize \textit{Airplane trip}}\\
        \tikz{\fill (0,0) rectangle (\cardwidth-\strippadding-\stripwidth-2*\textpadding,\ruleheight);}\\
        {\small Each player passes one microbe in play to the player on their left}\\
        {\small \small }
        \rule{4cm}{0.4pt}
        {\small \small \textit{Sharing is caring}}\\
        
    };
\end{tikzpicture}

&

\begin{tikzpicture}

	\draw[red,fill=red] (5.2,.9) circle (4ex);
	\draw[red,fill=green] (3.7,.9) circle (4ex); 
	\node at (3.7,0.9) {\LARGE \bfseries 0}; 
	\node at (5.2,0.9) {\LARGE \bfseries 0};
    \draw[rounded corners=\cardroundingradius] (0,0) rectangle (\cardwidth,\cardheight);
    \fill[pink,rounded corners=\striproundingradius] (\strippadding,\strippadding) rectangle (\strippadding+\stripwidth,\cardheight-\strippadding) node[rotate=90,above left,black,font=\stripfontsize] {Event \rotatebox[origin=c]{-90}{\ding{49}}};
    \node[text width=(\cardwidth-\strippadding-\stripwidth-2*\textpadding)*1cm,below right,inner sep=0] at (\strippadding+\stripwidth+\textpadding,\cardheight-\textpadding) 
    {   {\captionfontsize \textbf{}}\\ 
        {\textfontsize \textit{Raid the pharmacy}}\\
        \tikz{\fill (0,0) rectangle (\cardwidth-\strippadding-\stripwidth-2*\textpadding,\ruleheight);}\\
        {\small Search the deck for an antibiotic (tetracycline, kanamycin, or ampicillin).  Shuffle the deck afterwards}\\
        {\small \small }
        \rule{4cm}{0.4pt}
        {\small \small \textit{We're not suggesting you do this�}}\\
        
    };
\end{tikzpicture}

&

\begin{tikzpicture}

	\draw[red,fill=red] (5.2,.9) circle (4ex);
	\draw[red,fill=green] (3.7,.9) circle (4ex); 
	\node at (3.7,0.9) {\LARGE \bfseries 0}; 
	\node at (5.2,0.9) {\LARGE \bfseries 0};
    \draw[rounded corners=\cardroundingradius] (0,0) rectangle (\cardwidth,\cardheight);
    \fill[pink,rounded corners=\striproundingradius] (\strippadding,\strippadding) rectangle (\strippadding+\stripwidth,\cardheight-\strippadding) node[rotate=90,above left,black,font=\stripfontsize] {Event \rotatebox[origin=c]{-90}{\ding{49}}};
    \node[text width=(\cardwidth-\strippadding-\stripwidth-2*\textpadding)*1cm,below right,inner sep=0] at (\strippadding+\stripwidth+\textpadding,\cardheight-\textpadding) 
    {   {\captionfontsize \textbf{}}\\ 
        {\textfontsize \textit{Raid the pharmacy}}\\
        \tikz{\fill (0,0) rectangle (\cardwidth-\strippadding-\stripwidth-2*\textpadding,\ruleheight);}\\
        {\small Search the deck for an antibiotic (tetracycline, kanamycin, or ampicillin).  Shuffle the deck afterwards}\\
        {\small \small }
        \rule{4cm}{0.4pt}
        {\small \small \textit{We're not suggesting you do this�}}\\
        
    };
\end{tikzpicture}

\end{tabular}
\cleardoublepage\end{document}